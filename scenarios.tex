\begin{multicols}{2}
\section{Scenarios}
\settocdepth{subsection}
\renewcommand*{\thesubsection}{\Alph{subsection}}
\newcommand{\myparagraph}[1]{\paragraph{#1}\mbox{}\\}


Several scenarios are available for selection depending on desired
play time and/or player skill.

\begin{enumerate}[label=\Alph*]
\item Hanshin Railways: 2 to 3 hours, 2 to 4 players
\item Keihanshin Railways: 2 to 3 hours
\item 1890: 3 to 5 hours
\item 1890 Simplified: 3 to 5 hours
\end{enumerate}

\newcounter{scenario}
\setcounter{secnumdepth}{4}


\subsection{Hanshin Railways}

This scenario was designed to play as an 18XX introductory game. It is
designed to finish quickly with a small number of players, and almost
the same game phase structure. Experienced players can finish a game
in 2 to 3 hours.

Experience the battle of railways in Hanshin between Hanshin Railway,
Hankyu Railway, and JR.

In this scenario, 6-trains and D-trains are very unlikely to
appear. Once you get used to it, try playing the next `Keihanshin
Railways Scenario`.

\subsubsection{Number of players}

2 to 4 (2 to 3 recommended)

The two-player game is an austere experience whereas the three-player
game is more enjoyable. The four-player game is an unending
experience.

\subsubsection{Map}

The entire map west of the Yodo River (towards Kobe), including Osaka
Kita and Kyoto is used. As per the normal rules, track tiles may not be laid
pointing towards the Yodo River.

\subsubsection{Track tiles}

All usable (except for special tiles cities not included in the scenario).

\subsubsection{Trains}

\begin{tabular}{l|cccccccc}
Type & 2 & 2+2 & 3 & 3+3 & 4 & 5 & 6 & D \\ \hline
Count & 5 & 0 & 4 & 0 & 3 & 2 & 1 & 5
\end{tabular}

\subsubsection{Money}
The bank size is 6,000円. Set aside all 500円 and 1000円
banknotes for use after the bank breaks. Divide 1,500円 evenly
among players as starting capital.

\begin{tabular}{|l|c|c|c|}
\hline
Players & 2 & 3 & 4 \\
\hline
Start Money & 750 & 500 & 375 \\
\hline
\end{tabular}

\subsubsection{Phases}
Phase 1 to phase 6 are played as normal. However, the concept of the
first half and second half of phases 1 and 2 is not relevant.

\subsubsection{Certificate Limits}
Use the following certificate limits according to the number of players:

\begin{tabular}{|l|c|c|c|}
\hline
Players & 2 & 3 & 4 \\
\hline
Certificates & 16 & 12 & 10 \\
\hline
\end{tabular}

\subsubsection{Companies}
Due to the limited scope of the map, fewer companies are used. There
are 5 private companies, 1 minor company and 4 public companies. Late
private companies are not used, except in the case where Kobe Electric
Railway converts to a late private company. Also, at the start of the
game, the initial stock round will start in the following company
order. Keishin Railway and Osaka Municipal Electric Railway use the par
value and the dividend amount as shown below.

\begin{table*}
\begin{tabular}{lllll}
 & Company & Par/Revenue (円) & Type & Modified? \\
\hline
A & Arima Railway & 20/5 & Private & \\
B & Kobe Tramway & 40/10 & Private & \\
C & Keishin Railway & 60/15 & Private & X \\
D & Hanshin Tramway & 110/20 & Private & \\
E & Osaka Municipal Electric Tramway & 120/30 & Private  & X \\
F & Kobe Electric Railway & 100/- & Minor  & \\
\end{tabular}
\end{table*}

\myparagraph{Private Companies}

In this scenario, every private company can be purchased by a public
company. Until the private company is purchased by a public company or
closed, the hexes with tracks for these private companies can not have
track tiles laid on them. The rules and benefits for each private
company are as follows:

\begin{description}
\item[Arima Railway] 20円 / 5円

  Once purchased by a public company, you can place an additional tile
  on Arima in addition to the regular track tile placement. The track
  tile does not need to be connected. This power disappears when Arima
  Railway closes.

\item[Kobe City Tramway] 40円 / 10円

  There are no special features.

\item[Keishin Railway]  60円 / 15円

  There are no special features.

\item[Hanshin National Highway Orbit] 110円 / 20円

  Comes with one share of Hanshin Electric Railway.
\end{description}

\myparagraph{Late private companies}

Late private companies are not used except for the case of exchanging
the Kobe Electric Railway minor company for a late private company.

\myparagraph{Minor Companies}
\begin{description}
\item[Kobe Electric Railway] \hfill

  Home: Tanigami \hfill Par Value: 100円  \hfill Starting Capital: 200円

  At any point during it's operating round turn, Kobe Electric Railway
  may convert to a late private company. After converting, station
  tokens are remvoed and all assets are returned to the bank. Although
  the Kobe Electric Railway late private company certificate does not
  count toward the certificate limit, its face value is 0円. From
  this point onwards, any public company may place a station token in
  the city previously occupied by Kobe Electric Railway's token
  (Tanigami).

\end{description}

\myparagraph{Public companies}
It is said that the development of the game is driven by public
companies. In order to ease play, no special rules are used for this
scenario.
\begin{description}
\item[Hanshin (Hanshin Electric Railway)] \hfill

  Home: Nishitomi \hfill Stations: 3

  The company is started when 4 shares are purchased because 1 share
  of Hanshin stock comes with the Hanshin Tramway private company.

  The Hanshin stock attached to the Hanshin Tramway can not be sold
  until the president stock of Hanshin is purchased.

\item[Hankyu (Hankyu Railways)] \hfill
Home: Toyonaka \hfill Stations: 4

  No special rules.

\item[JR (Japan Railways)] \hfill \label{JR}
Homes: Osaka Kita, Kyoto, Nara, Kobe \hfill Stations: 6

JR is treated as a regular public company except for the following rules:
\begin{itemize}
\item The par price must be 100円.
\item JR has a higher train limit than other public companies.

\begin{tabular}{ll}
  Phase & Limit \\
  \hline
  1 &  6 \\
  2-3 & 4 \\
  4-6 & 3
\end{tabular}
\item JR can only distribute operating revenue as half dividends. Half
  of the revenue (rounded up to the nearest 10円) is paid directly
  to the company treasury, the remaining half is distributed among
  shareholders. Revenue can also be withheld as normal.
\item In game phases 1 to 3, JR may place or upgrade two track tiles in
  two different hexes. Performing two tile actions on the same hex is
  not allowed.
\end{itemize}

\item[Sanyo (Sanyo Electric Railway)] Home: Akashi Stations: 2

No special rules.
\end{description}
\subsubsection{End of the game}

The game will end if any of the following conditions are met:
\begin{enumerate}
\item The bank breaks

\item Player bankruptcy

\item After 10 game turns.
\end{enumerate}

\subsubsection{Variants}

Please try to provide some variants. Also, try using the 1890 scenario
variant if you are used to it.

\myparagraph{Specialization of public companies}
Incorporate and play by the special rules of each railway company from the
1890 scenario.

\myparagraph{Start capital}
Reduce the amount of money held by the first player. The amount of
money should be decided between players.

\myparagraph{JR's rule change}
Play without special rules for JR (extra track actions, half
dividends, fixed par price).

\myparagraph{Number of game turns}
The game ends after 8 game turns instead of 10.

\newpage
\subsection{Keihanshin Railways}

This scenario is an extension to the Hanshin Railways
scenarios. Frequently, the 6-trains will not come out.

\subsubsection{Number of players}

2 to 4 players

\subsubsection{Map}

The entire map west of the Yodo River, and also the hexes directly
adjacent to the east (right) of the Yodo River. Osaka city includes
Osaka Kita and Osaka Higashi. The hex to the east of Kyoto is also
used.

\subsubsection{Track tiles}

All usable (except for unused special locations).

\subsubsection{Trains}
\begin{tabular}{l|cccccccc}
Type & 2 & 2+2 & 3 & 3+3 & 4 & 5 & 6 & D \\ \hline
Count & 5 & 0 & 4 & 0 & 3 & 2 & 1 & 6
\end{tabular}

\subsubsection{Money}
The bank size is 6,000円. Set aside all 500円 and 1000円 for
use after the bank breaks.

Divide 1,500円 evenly among players as starting capital.

\begin{tabular}{|l|c|c|c|}
\hline
Players & 2 & 3 & 4 \\
\hline
Start Money & 750 & 500 & 375 \\
\hline
\end{tabular}

\subsubsection{Phases}
Phase 1 to phase 6 are played as normal. However, the concept of the
first half and second half of phases 1 and 2 is not relevant.

\subsubsection{Certificate Limit}
Use the following certificate limits according to the number of players:

\begin{tabular}{|l|c|c|c|}
\hline
Players & 2 & 3 & 4 \\
\hline
Certificates & 16 & 12 & 10 \\
\hline
\end{tabular}

\subsubsection{Companies}
Due to the limited scope of the map, fewer companies are used. There
are 5 private companies, 1 minor company and 5 public companies. Late
private companies are not used, except in the case where Kobe Electric
Railway converts to a late private company. Also, at the start of the
game, the initial stock round will start in the following company
order. Osaka Municipal Electric Railway uses the modified par
value and the dividend amount as shown below.

\begin{tabular}{lp{2cm}llc}
 & Company & Par/Revenue (円) & Type & Modified? \\
\hline
A & Arima Railway & 20/5 & Private  & \\
B & Kobe Tramway & 40/10 & Private  & \\
C & Hanshin Tramway & 110/20 & Private  & \\
D & Osaka Municipal Electric Tramway & 120/ 30 & Private  & X \\
E & Keishin Railway & 160/25 & Private & \\
F & Kobe Electric Railway & 100/- & Minor & \\
\end{tabular}

\subsubsection{Private Companies}

Same as the Hanshin Railways scenario. However, the following changes will
be made to Keishin Railway.

\begin{description}
\item[Keishin Railway]  160円/ 25円

Comes with one regular share of Keihan Electric Railway.
\end{description}

\subsubsection{Minor companies}
Same as the Hanshin Railways scenario

\subsubsection{Late private companies}
Same as the Hanshin Railways scenario

\subsubsection{Public companies}
Same as Hanshin Railways scenario, with the addition of Keihan Electric Railway.

\begin{description}
\item[Keihan (Keihan Electric Railway)] \hfill

Homee: Hirakata \hfill Stations: 3

The company will be started if 4 shares are purchased from the initial
offer, since the Keishin Railway private company comes with one share of
Keihan Electric Railway. No shares of Keihan Electric Railway may be
sold until its president's certificate has been purchased.
\end{description}

\subsubsection{End of the game}
The game ends if any of the following conditions are met,
\begin{enumerate}
\item The bank breaks
\item Player bankruptcy
\item After 8 game turns
\end{enumerate}

\subsubsection{B10 Variants}
Same with the Hanshin Railways scenario

\newpage
\subsection{1890}

This is the full 1890 scenario.

If you are playing this game for the first time, the Hanshin Railways,
Keihanshin Railways, or 1890 Simplified scenario may be a better
introduction.

\subsubsection{Number of players}

The 1890 scenario is designed to be played with 2 to 7 players. A two
player game is very difficult and it will be difficult for a player
who makes two consecutive mistakes to be victorious. Three and four
players can enjoy various bargains if the player is familiar with the
rules. If you have fun as a multiplayer game, I recommend 5-6 play. It
will be the most standard play.

\subsubsection{Map}

The entire map is used.

\subsubsection{Track tiles}

All are available.

\subsubsection{Trains}
\begin{tabular}{l|cccccccc}
Type & 2 & 2+2 & 3 & 3+3 & 4 & 5 & 6 & D \\ \hline
Count & 9 & 3 & 5 & 2 & 4 & 3 & 2 & 6
\end{tabular}

\subsubsection{Money}
The bank size is 12,500円. Set aside 12 500円 and 12 1000円
banknotes for use after the bank breaks. Divide 2,520円 evenly
among players as starting capital.

\begin{tabular}{l|llllll}
Players & 2 & 3 & 4 & 5 & 6 & 7 \\ \hline
Money(円) & 1250 & 840 & 630 & 504 & 420 & 360
\end{tabular}

\subsubsection{Phases}

Phases 1 to phase 6 are played as normal.

\subsubsection{Certificate Limits}
Use the following certificate limits according to the number of players:

\begin{tabular}{l|llllll}
Players & 2 & 3 & 4 & 5 & 6 & 7\\
\hline
Certificates & 26 & 18 & 15 & 13 & 11 & 10 \\
\end{tabular}

\subsubsection{Companies}

In the 1890 scenario, 6 private companies, 5 minor companies, 4 late
private companies, and all 8 public companies are in use. At the start
of the game, start the initial stock round in the following company
order:

\begin{description}
\item[Private companies] \hfill

\begin{tabular}{lp{3cm}l}
 & Company & Par (円) \\
\hline
A & Arima Railway & 20\\
B & Kobe Tramway & 40\\
C & Hankai Tramway & 70 \\
D & Hanshin Tramway & 110\\
E & Keishin Railway & 160\\
F & Osaka Municipal Electric Railway & 120
\end{tabular}
\item[Minor companies] \hfill

\begin{tabular}{lll}
 & Company & Par (円) \\
\hline
G & Kanan Railway & 100\\
H & Osaka Electric Railroad & 200\\
I & Osaka Railway & 100 \\
J & Nara Electric Railway & 160\\
K & Kobe Electric Railway & 100
\end{tabular}
\end{description}

\myparagraph{Private companies}\label{privates}
Private companies block tile lays on the map hexes where their track
is. They also carry certain benefits and abilities which are explained
as follows:

\begin{description}
\item[Arima Railway] \hfill 20円/5円

  The public company which purchases Arima Railway may place a track
  tile in Arima in addition to its regular track tile placement. The
  tile does not need to be connected.

\item[Kobe Tramway] \hfill 40円/10円

  No special rules

\item[Hankai Tramway] \hfill 70円/ 15(5)円

  Does not close in phase 4. However, its revenue is reduced to 5円, it may
  no longer be purchased by a public company, and continues to count
  as one certificate toward the certificate limit.

\item[Hanshin Tramway] \hfill 110円/20円

Comes with one regular share of Hanshin Electric Railway.

\item[Keishin Railway] \hfill 160円/25円

Comes with one regular share Keihan Electric Railway.

\item[Osaka Municipal Electric Railway] \hfill 220円/ 40円 \label{osaka-municipal}

  The purchasing player immediate sets the par price of the Osaka
  Municipal Subway public company and receives the president's
  certificate. The par value of Osaka Muncipal Electric Railway then
  becomes 0円.

  During Osaka Muncipal Subway's first operating round turn, it may
  upgrade one of Osaka Kita, Osaka Higashi, or Osaka Nishi for free.

  Osaka Municipal Electric Railway closes when Osaka Municipal Subway
  buys a train. Even if Osaka Municipal Subway has started, but is
  trainless, Osaka Municipal Electric Railway will continue to stay
  open, but the public company's stock price will fall due to
  non-payment of dividends.
\end{description}

\myparagraph{Late private companies}
This scenario includes four late private companies. Their rules and
benefits are as follows:

\begin{description}
\item[Keifuku Electric Railroad] \hfill 200円/40円

  If Keihan has a station token in Kyoto, it will receive 40円 to
  its treasury each operating round.

\item[Kobe Rapid Transit ​​Railway] \hfill 240円/? \label{kobe}

  Kobe Rapid Transit Railway is a special railway company with no
  trains. There is no fixed income and it earns no revenue from its
  operation. Although it has a station token in Kobe, there is no
  obligation to own trains.

  Kobe Rapid Transit Railway is started at the moment of purchase by a
  player. Station tokens other than JR in Kobe will be returned to the
  owning public companies, the number 6 turn token will be placed in
  Kobe as a station token of Kobe Rapid Transit Railway. This station
  token occupies a city space like a regular station tokens and
  affects train operations in the same way. From this point on, if
  there is a free space in Kobe, any public company may place their
  station tokens in Kobe as usual.

  Public companies may ignore the existence of the Kobe Rapid Transit
  Railway station token by paying a 100円 fee to the bank. This
  payment is considered to be the company's one station token
  placement per operating round turn, but does not consume one of its
  station tokens. This does not apply to JR because it already has a
  station token in Kobe. Furthermore, Osaka Municipal Subway cannot
  use this privilege.

  Any public company that previously had its token in Kobe is
  considered to have obtained the privilege to ignore the Kobe Rapid
  Transit Railway token.

  Every time a railway company that does not have a station token in
  Kobe counts Kobe in its train route for revenue (including those
  companies which have obtained the privilege described above), Kobe
  Rapid Transit Railway pays half the value of Kobe as dividend,
  without changing the revenue for the other railway company.

\item[Kita-Osaka Kyuko Railway] \hfill 280円/60円

  In the operating round following the purchase of the first 6-train,
  pays a one-time dividend of 100円 in addition to the usual 60円 dividend.

\item[Semboku Rapid Railway] \hfill 320円/70円

  Any public company with a station token in Sakai receives 40円
  into its treasury each operating round.
\end{description}

\myparagraph{Minor companies} \label{sec:minor-companies}
This scenario includes five minor companies. The number printed on the
minor company's charter is its operating order, and the corresponding
turn order token is used as its station token. Minor companies 2 and 3
operate out of the same station token in Osaka Higashi.

The minor companies 1 to 4 will convert or merge into the Kinki Nippon
Railways public company at certain points in the game. This conversion
or merger may be declared at any point during the minor company's
operating turn.

Each minor company owned by a player counts towards the player's
certificate limit. The rules for each minor company are as follows:

\begin{description}

\item[1 Kanan Railway] \hfill

Home: Norohara \\
Par price: 100円 \\
Start Capital: 100円

Beginning in phase 2, merger with Kinki Nippon Railway is
possible. The merger with Kinki is forced in phase 3. Kanan Railways
is closed and exchanged for a 10\% regular share of Kinki. Kinki
receives half of Kanan Railway treasury (rounded up) and its
trains. The remainder of the treasury is given to the owner of Kanan
Railway.

\item[2 Osaka Electric Railroad] \hfill

Home: Osaka Higashi \\
Par price: 200円 \\
Start capital: 200円

Osaka Electric Railroad is the predecessor of the Kinki Nippon
Railways public company. When purchased in the initial stock round,
the owner of Osaka Electric Railroad immediately sets the par price
for Kinki.

Beginning in phase 2, Osaka Electric Railroad may convert into Kinki
Nippon Railways. The conversion into Kinki is forced when the second
half of phase 2 begins. Transfer all of Osaka Electric Railroad's
assets to Kinki. The owner of Osaka Electric Railroad receives the
20\% president's certificate of Kinki in exchange, and Kinki
immediately floats.

\item[3 Osaka Railway] \hfill

Home: Osaka Higashi \\
Par price: 100 \\
Start capital: 100

When Kinki Nippon Railways floats, Osaka Railway is forced to
merge. All assets are transferred to Kinki, and the owner of Osaka
Railway receives a 10\% regular share of Kinki in exchange.

\item[4 Nara Electric Railway] \hfill

Home: Kyoto, Nara \\
Par price: 160円 \\
Start capital: 320円

May merge into Kinki Nippon Railways from the beginning of phase 4. Is
forced to merge at the beginning of phase 5. All assets are
transferred to Kinki and the owner of Nara Electric Railway receives 2
regular shares of Kinki in exchange.

If Kinki already has its station token in the same city as Nara
Electric Railway's tokens, the minor company's token is
removed. Otherwise, it is replaced with a Kinki station token.

\item[5 Kobe Electric Railway]\hfill

  Home: Tanigami \\
  Par price: 100円 \\
  Start capital: 200円

  At any point during its operating round turn, Kobe Electric Railway
  may convert to a late private company. After converting, station
  tokens are remvoed and all assets are returned to the bank. Although
  the Kobe Electric Railway late private company certificate does not
  count toward the certificate limit, its face value is 0円. From
  this point onwards, any public company may place a station token in
  the city previously occupied by Kobe Electric Railway's token
  (Tanigami).
\end{description}

\myparagraph{Public companies}
The 1890 scenario includes eight public companies. Each company has
specific rules:

\begin{description}
\item[JR (Japan National Railways)] \hfill

Homes: Osaka Kita, Kyoto, Nara, Kobe \\
Stations: 6

Historically, JR was a state-owned railway until phase 6, and it was
operated with the national interest under consideration. In this game,
JR represents the JR-West subsidiary of the Japan Railways group. JR
is played as a normal public company except for the following:
\begin{itemize}
\item The par price for JR is 100円

\item JR has a higher train limit than other public companies.

Phase 1: 6 trains\\
Phase 2-3: 4 trains\\
Phase 4-6: 3 trains

\item When distributing operating revenue, JR pays half
  dividends. The first half of the revenue (rounded up) is paid to the
  company treasury. The remainder of the revenue is then distributed
  among shareholders. Revenue may also be withheld as usual.

\item In phases 1 to 3, JR can lay or upgrade 2 track tiles per
  operating round in different hexes. Performing two tile actions in
  the same hex in one operating round is prohibited.
\end{itemize}

\item[Osaka Subway (Osaka Municipal Subway)] \hfill \label{osaka-subway}

Home: Osaka Nishi \hfill Stations: 1

The par price for Osaka Municipal Subway is set immediately when the
Osaka Municipal Electric Railroad private company is purchased in the
initial stock round.

\subparagraph{Subway service}
After phase 4, subway service is established in the city of
Osaka. Osaka Subway may ignore other companies' station tokens in the
Osaka tiles (Osaka Kita, Osaka Higashi, Osaka Nishi, and Osaka Minami)
if they have been upgraded to brown.

\subparagraph{Special tile placement}
During its first operating turn, Osaka Subway can place one tile in
Osaka free of charge.

\item[Keihan (Keihan Electric Railway)] \hfill

Home: Hirakata \hfill Stations: 3

Keihan will be float when 4 shares have been bought from the
initial offer because the Keishin Railway private company comes with
one regular share of Keihan.

The Keihan share attached to Keishin Railway may not be sold until the
president's certificate of Keihan has been purchased.

\item[Nankai (Nankai Electric Railway)] \hfill

Home: Osaka Minami \hfill Stations: 3

No special rules

\item[Kinki (Kinki Nippon Railways)] \hfill
\label{kinki}

Home: Osaka Higashi (Kashiwara, Kyoto, Nara) \\
Stations: 6

Kinki Nippon Railways was historically formed through the merger of a
large number of smaller railway companies. In this scenario, the Osaka
Electric Railroad, Osaka Railway, Kanan Railway, and Nara Electric
Railway minor companies will merge to form Kinki.

The par price of Kinki share certificates will be decided immediately
by the player who purchased Osaka Electric Railroad in the initial
stock round.

Each minor company will be replaced by one or two Kintetsu share
certificates upon merger with Kintetsu. Moreover, the president's
certificate of Kinki is reserved for Osaka Electric Railroad. These
trade-in share certificates are reserved by the bank until then and can not
be bought or sold until the merger or conversion. In other words, only
4 Kinki stock certificates can be purchased from the initial
offer. These four stock certificates can be bought and sold from the
first stock round as usual, but they will not earn any income until
Kinki begins operations.

Each minor company may declare merger or conversion into Kinki, in
between company operations. The process and timing for merger and
conversion is described in \autoref{sec:minor-companies}

Kinki Nippon Railway forms when Osaka Electric Railway declares that
it will convert. Then, the Osaka Railway minor company is immediately
merged into Kinki. Kanan Railway may also choose to merge at this
time. Kinki immediately floats, even if none of its shares have been
purchased from the initial offer. It now receives additional starting
capital equal to 4 times its par price.

Immediately after Kinki floats, it performs a special operating round
out of turn order. During this special operating round, the share
price of Kinki does not drop due to non-payment of dividends. If
dividends are paid, the share price increases as normal. Kinki does
not operate again during the round in which it performed this special
operating round.

Kinki's stock price will not rise until Nara Electric Railway merges,
as the stock certificates reserved for mergers do not count as being
sold.

If Kinki closes before all minor companies have merged, the remaining
minor companies are unable to merge with Kinki and will close when they would
otherwise be forced to merge.

\item[Sanyo (Sanyo Electric Railway)] \hfill

Home: Akashi \hfill Stations: 2

No special rules

\item[Hanshin (Hanshin Electric Railway)] \hfill

Home: Nishinomiya \hfill Stations: 3

Hanshin floats when 4 shares are purchased, since the Hanshin Tramway
private company comes with 1 share of Hanshin stock.

Until the Hanshin president's certificate is purchased, the share
attached to Hanshin Tramway cannot be sold.

\subparagraph{Hanshin Tigers}
Hanshin Electric Railway receives bonus revenue due to the popularity
of the Hanshin Tigers baseball team. After Nishinomiya is upgraded to
brown, if Hanshin operates routes which include Nishinomiya, increase
operating revenue by 10円, and also add 10円 to the Hanshin company
treasury from the bank.

\item[Hankyu (Hankyu Railways)] \hfill

Home: Toyonaka \hfill
Stations: 4

\subparagraph{Takarazuka Revue}
Hankyu profits from popularity of the Takarazuka Revue musical
troupe. If Hankyu has a station token in Takarazuka, it receives
40円 to its treasury each operating round.


\subparagraph{Hankyu Land Commercial Code}
Every time Hankyu lays a yellow track tile, it receives 10円 into the
company treasury.

\end{description}

\subsubsection{End of the game}

The game will end if any of the following conditions are met:
\begin{enumerate}
\item The bank breaks

\item Player bankruptcy

\item After 10 game turns.
\end{enumerate}

\subsubsection{Variants}
These variants can be used to change the evolution of the game. Some
variants will disrupt game balance and change the experience
significantly.

\myparagraph{Hidden personal cash}
The amount of player cash is secret. Players do not need to respond to
inquiries about their cash holdings.

\myparagraph{Hiding of train purchase price}
The buying and selling of trains between companies must be made public
to other players, but it is not necessary to make their prices public.

\myparagraph{Change D-trains to 8-trains}
D-trains can visit only 8 cities in their route. All other properties,
including the ability to trade in a 4-, 5-, or 6-train, are the same.

\myparagraph{Forced closure of minor companies}
At the beginning of phase 5, Kobe Electric Railway is forced to
convert to a late private company.

\myparagraph{End of game when highest stock price is reached}
The game is over when the stock price of a public company reaches 400円.
If the price reached 400円 during the operating round, the game ends
immediately after the company's operating turn. If the prices reaches
400円 at the end of a stock round, the game ends immediately.

\myparagraph{Kobe Rapid Transit Revenue Control}
Reduce the value of Kobe to 10円 in yellow, 15円 in green, and 20円 in brown.

\myparagraph{Late private company abilities}
Remove all late private company special abilities. Kobe Rapid Transit
Railway pays a fixed dividend of 50円.

\myparagraph{Tile placement restrictions}
Brown tile \#78 may only be placed in Nara or a city bordering the sea
or the Yodo river.

\myparagraph{Public company floatation}
This variant changes the rules for floating public companies, making
them easier to start. Incorporating this variant will produce a more
aggressive game experience.

When a company would begin its very first operating turn, check the
number of 10\% shares which have been sold from the initial
offering. If this number is greater than or equal to the rank of the
train currently for sale from the bank, then the company floats and
may operate. Otherwise, the company is considered not to have floated,
and will not take an operating turn. At the beginning of the game,
only the 20\% president's certificate needs to be sold in order for a
company to float and operate. Later in the game, if for example the
current train available for sale is a 3T or 3+3T, then a public
company must have sold at least 3 shares from the initial offer in
order float and operate.

Instead of receiving ten times the par value when floating, public
companies receive money as shares are sold from the initial offer. The
shares of Hanshin, Keihan, and Osaka Subway that are attached to
private companies are considered to be already sold, and the
respective public companies will begin the game with those funds in
their treasuries.

Additionally:
\begin{itemize}

\item After phase 5, public companies will be started using the normal
  rules. 50\% of shares must be sold from the initial offer to float,
  and the company receives 10 times the par value to its treasury.

\item JR does not float unless 5 shares (50\%) are purchased
  from the initial offer

\item Kinki Nippon Railway is started as normal, when Osaka Electric
  Railway converts. If the non-reserved shares of Kinki are bought
  from the initial offer before the company is formed, the money from
  those sales is still added to the Kinki treasury.

\end{itemize}

\myparagraph{Privatization of JR}
Historically, JR transformed from a national railway to a private
railway company. After phase 6, JR will pay full dividends instead of
half.

\myparagraph{Kinki as a Normal Company} \label{kinki-variant}
While the treatment of Kinki Nippon Railways in 1890 aims to simulate
its historical development, it also creates extra rules burdens. In
this variant, minor companies 1-4 are eliminated, and Kinki is started
like a normal public company.

In addition to 2 regular station tokens, Kinki will have 4 station
tokens that are automatically placed at the beginning of the following
game phases:

\begin{description}
\item[Phase 1] Osaka Higashi
\item[Phase 2] Kashiwara
\item[Phase 3] Nara
\item[Phase 4] Kyoto
\end{description}

Automatically placed station tokens are placed at the very beginning
of each operating round, at no cost. These tokens are treated as home
tokens, and reserve a space in their cities.  Follow normal token
placement rules for the automatic station tokens: connectivity and a
free city space is required. If possible, multiple automatic tokens
may be placed at once.

When using this variant, remove three 2-trains and all 3+3-trains.

\myparagraph{Hanshin Tigers}
This variant requires one six-sided die. After Nishinomiya is upgraded
to brown, instead of receiving a fixed amount, Hanshin Electric
Railway will receive a variable bonus if its routes include
Nishinomiya.

Each operating turn in which Hanshin runs to the brown Nishinominya
tile, roll the die. The value of Nishinomiya will be determined by the
result and the following table:

\begin{tabular}{l|llllll}
Die Roll & 1 & 2 & 3 & 4 & 5 & 6 \\
\hline
Value & 100 & 60 & 50 & 40 & 40 & 40 \\
\end{tabular}

\myparagraph{Great Hanshin Earthquake}
This variant simulates the Great Hanshin Earthquake of January 1995,
which caused major damage to the Hanshin region. In particular,
Nishinomiya, Ashiya, and Kobe suffered heavy damage, and urban
functions were destroyed in the city centers. Building collapses and
fires caused many to lose their homes and many of the affected people
still live in refugee housing. Many industries were disrupted due to
the ensuing financial instability and economic recession. Despite
signficant time and money spent for recovery, the effects of this
disaster are still seen today.

Label the reverse side of one D-train with the word ``Earthquake'',
and shuffle it with the other D-trains during game setup. The Great
Hanshin Earthquake will occur when this labeled D-train is purchased,
unless the bank has already broken at that point.

If the earthquake occurs, then until the next stock round Hanshin
Railway's income will be 0, and the incomes for Hankyu, Sanyo, and JR
will be reduced by half. During this time, stock prices will be frozen
and will not move.

Kobe Rapid Transit Railway is closed, and the station tokens in Kobe
and Ashiya are removed and returned to their owning companies. The
Nishinomiya tile is downgraded one level. Tokens in Itami are flipped
face down. After the next stock round, companies may pay 100円 to flip
them back face up.

Charity Payments: Each player pays the following amount to the bank by
the end of the next stock round.

\begin{align*}
&(\textrm{\# D-trains already sold} \times \textrm{100円}) + \\
&(\textrm{\# of shares held} \times \textrm{200円})
\end{align*}

This amount is calculated at the time the earthquake occurs. If cash
on hand is insufficient to make the payment, then shares must be sold to
make up the difference.


\myparagraph{Second World War Air Raids}
At the end of World War II, the United States military conducted a
sustained campaign of air raids in cities around Japan's four
largest industrial areas. These bombings resulted in many civilian
deaths and impeded economic recovery after the war. The Hanshin region
was particularly impacted as it formed a large part of Japan's
industrial base, Osaka, Kobe, and Amagasaki were targeted for bombing.

After the last 3+3-train is purchased, the air raids will take
place. Discard all green tiles from Kobe, Ashiya, Nishinomiya,
Amagasaki, Osaka Kita, Osaka Nishi, Osaka Higashi, and Osaka Minami,
as well as surrounding hexes. Yellow tiles will not be
discarded. Green track tiles that were removed from empty map hexes
must be laid again starting with yellow, and repaying any terrain costs.

After green tiles have been removed, station tokens are
replaced. Tokens may not be replaced into spaces reserved for
unstarted companies. If a city contains station tokens from multiple
companies and there are no longer enough token spaces, then the tokens
are placed in operating order beginning with the company after the one
that purchased the last 3+3-train. Remaining tokens will be placed on
the tile, as a reserved token. When the tile is upgraded creating a
new token space, the reserved token will automatically fill that
space. At any time, reserved tokens may be removed to be used
elsewhere as a normal 100円 station token.

\newpage

\subsection{1890 Simplified}

This is a tactical version of the 1890 scenario. Special rules for
companies are removed so that players can become accustomed to the
game.

\subsubsection{Number of Players}
2-7 players.

\subsubsection{Map}
The entire map is used.

\subsubsection{Track tiles}
All are available.

\subsubsection{Trains}

\begin{tabular}{l|cccccccc}
Type & 2 & 2+2 & 3 & 3+3 & 4 & 5 & 6 & D \\ \hline
Count & 6 & 3 & 5 & 0 & 4 & 3 & 2 & 6
\end{tabular}

\subsubsection{Money}
The bank size is 12,500円. Set aside 12 500円 and 12 1000円
banknotes for use after the bank breaks. Divide 2,520円 evenly
among players as starting capital.

\begin{tabular}{l|cccccccc}
Players & 2 & 3 & 4 & 5 & 6 & 7 \\ \hline
Money(円) & 1250 & 840 & 630 & 504 & 420 & 360
\end{tabular}

\subsubsection{Phases}

Phases 1 to phase 6 are played as normal.

\subsubsection{Certificate Limits}
Use the following certificate limits according to the number of players:

\begin{tabular}{l|cccccccc}
Players & 2 & 3 & 4 & 5 & 6 & 7\\
\hline
Certificates & 26 & 18 & 15 & 13 & 11 & 10 \\
\end{tabular}
\columnbreak
\subsubsection{Companies}

This scenario includes, 6 private companies, and 1 minor company, and
all 8 public companies are in use. No late private companies are
used. At the start of the game, start the initial stock round in the
following company order:

\begin{description}
\item[Private companies] \hfill

\begin{tabular}{lp{4cm}l}
 & Company & Par (円) \\
\hline
A & Arima Railway & 20\\
B & Kobe Tramway & 40\\
C & Hankai Tramway & 70 \\
D & Hanshin Tramway & 110\\
E & Keishin Railway & 160\\
F & Osaka Municipal Electric Railway & 120
\end{tabular}
\item[Minor companies] \hfill

\begin{tabular}{lp{4cm}l}
 & Company & Par (円) \\ \hline
K & Kobe Electric Railway & 100
\end{tabular}
\end{description}

\myparagraph{Private Companies}
Same as the full 1890 scenario.

\myparagraph{Late Private Companies}
Not used.

\myparagraph{Minor Companies}
\begin{description}
\item[5 Kobe Electric Railway]\hfill

  Home: Tanigami \\
  Par price: 100円 \\
  Start capital: 200円

  At any point during its operating round turn, Kobe Electric Railway
  may convert to a late private company. After converting, station
  tokens are remvoed and all assets are returned to the bank. Although
  the Kobe Electric Railway late private company certificate does not
  count toward the certificate limit, its face value is 0円. From
  this point onwards, any public company may place a station token in
  the city previously occupied by Kobe Electric Railway's token
  (Tanigami).
\end{description}

\myparagraph{Public Companies}
The 1890 Simplified scenario includes eight public companies. Some
companies have special rules for floating or operation:
\begin{description}
\item[Kinki (Kinki Nippon Railways)] \hfill

Home: Osaka Higashi (Kashiwara, Kyoto, Nara) \\
Stations: 6

Use the Kinki as a Normal Company variant from the 1890 scenario
(\autoref{kinki-variant}).

\item[JR (Japan National Railways)]\hfill

Homes: Osaka Kita, Kyoto, Nara, Kobe \\
Stations: 6

Historically, JR was a state-owned railway until phase 6, and it was
operated with the national interest under consideration. In this game,
JR represents the JR-West subsidiary of the Japan Railways group. JR
is played as a normal public company except for the following:
\begin{itemize}
\item The par price for JR is 100円

\item JR has a higher train limit than other public companies.

Phase 1: 6 trains\\
Phase 2-3: 4 trains\\
Phase 4-6: 3 trains

\item When distributing operating revenue, JR pays half
  dividends. The first half of the revenue (rounded up) is paid to the
  company treasury. The remainder of the revenue is then distributed
  among shareholders. Revenue may also be withheld as usual.

\item In phases 1 to 3, JR can lay or upgrade 2 track tiles per
  operating round in different hexes. Performing two tile actions in
  the same hex in one operating round is prohibited.
\end{itemize}

\item[Osaka Subway (Osaka Municipal Subway)] \label{osaka-subway}

Home: Osaka Nishi \hfill Stations: 1

The par price for Osaka Municipal Subway is set immediately when the
Osaka Municipal Electric Railroad private company is purchased in the
initial stock round.

\subparagraph{Subway service}
After phase 4, subway service is established in the city of
Osaka. Osaka Subway may ignore other companies' station tokens in the
Osaka tiles (Osaka Kita, Osaka Higashi, Osaka Nishi, and Osaka Minami)
if they have been upgraded to brown.

\subparagraph{Special tile placement}
During its first operating turn, Osaka Subway can place one tile in
Osaka free of charge.

\item[Keihan (Keihan Electric Railway)]\hfill

Home: Hirakata \hfill Stations: 3

Keihan floats when 4 shares have been bought from the
initial offer because the Keishin Railway private company comes with
1 share of Keihan stock.

The Keihan share attached to Keishin Railway may not be sold until the
president's certificate of Keihan has been purchased.

\item[Nankai (Nankai Electric Railway)]\hfill

Home: Osaka Minami \hfill Stations: 3

No special rules.

\item[Hanshin (Hanshin Electric Railway)]\hfill

Home: Nishinomiya \hfill Stations: 3

Hanshin floats when 4 shares have been bought from the initial offer
because the Hanshin Tramway private company comes with 1 share of
Hanshin stock.

The Hanshin share attached to Hanshin Tramway may not be sold until
the president's certificate of Hanshin has been purchased.

\item[Hankyu (Hankyu Railways)]\hfill

Home: Toyonaka \hfill
Stations: 4

No special rules.
\end{description}

\subsubsection{End of the game}

The game will end if any of the following conditions are met:
\begin{enumerate}
\item The bank breaks

\item Player bankruptcy

\item After 10 game turns.
\end{enumerate}

\end{multicols}

%%% Local Variables:
%%% mode: latex
%%% TeX-master: "1890rules-en-translation"
%%% End:
