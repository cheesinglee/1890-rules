\section{Designer's Notes}
{\large Shinichi Takasaki (高崎真一)}
\begin{multicols}{2}
It was around the early summer of 1996 that I started making 1830 type
games with the idea of ​​using familiar themes of Japan and railways in
the Kinki region.

The first step was to decide the scale of the map.

The Kinki region's railway network includes Osaka, Kyoto, Kobe, Nara,
and Wakayama. There are also private railways in Shiga Prefecture and
Himeji. JR's rail network also includes all of them. Therefore, we
decided to include the 5 major private railway companies (Hanshin,
Hankyu, Keihan, Kinki, and Nankai). However, there are significant
differences between Kinki and the other four companies.

In the end, I wanted to create a scale that would allow me to
comfortably express the distance between Kyoto and Osaka, between Kobe
and Osaka, and between Wakayama and Osaka. I decided to set the
distnace between Kyoto and Osaka (about 34 km) to 5 hexes. In other
words, the distance between hexes is about 7 km.

The next issue in map design was how to represent Osaka. Osaka is too
big to be represented by a single hex, and there was also the issue of
Osaka's revenue value. Osaka, which has twice the population of Kyoto
or Kobe, must become a source of income for each railway. This problem
was solved by representing Osaka with four hexes, which would also
allow the Osaka Loop Line to be represented.

I wanted to have a total of eight public companies, so I needed to
select three more in addition to the five major companies I had
already decided to use.

JR is a clear choice, but representing its large scale is a
problem. Historically, JR was formed through the merger of various
private and provincial railways into the national railway. I
considered a merger process like in 1856, it was not a good fit due to
differences in the timings and reasons for the mergers. In the end, I
decided on using the combination of multiple tile lays with multiple
start tokens to represent the parallel track development in multiple
locations.

The seventh company was the Osaka Subway. The Osaka Subway has a
rather long rail network, but there is not much development into the
suburbs. This was represented by its sole station token in Osaka Nishi.

I decided on Sanyo Electric Railway for the eight company. Its
addition required that the map be spread westward from Kobe. This was
solved by expanding two hexes to the west and making Himeji an
off-board area.

I was bothered by Kinki's treatment. Compared to other railway
companies, it was too large because I wanted to represent its history
as the merger of several smaller companies. Even ignoring its routes
near Nagoya and considering only its routes near Osaka, we must deal
with several mergers: Daiki, Daitetsu, Sangu Express, Nara Electric,
and so on. Therefore, I decided to form Kinki like in 1835 or 2038. By
changing the merger timings, I was able to represent how some
companies did not join until the 1960s, such as Nara Electric. As a
result, the formation of Kinki is an important theme of the game.

The next issue was the choice of private companies. At first I
considered selecting railways with values of 20, 40, 70, 110, 160, and
220 as in 1830, but could not decide on a good set. In general there
are too many private railway companies in Kansai; it is called the
``private railway kingdom'' for good reason.

\emph{What was once famous has been erased by the waves of time...}


One such thing is the tram. Osaka, Kyoto, and Kobe streetcars, Hanshin
Tramway, Keishin Electric Railway ... Of course, Hanshin Electric
Railway and Keifuku Electric Railway still operate today.

First of all, you can think of Osaka Municipal Electric Railway as the
B\&O, where Osaka Subway is started instead. Next, Keishin Electric
Railway, which has recently been lost due to the introduction of the
Kyoto tram, is a high-value line. (In fact, it was under the control
of Keihan in 1996 when I started designing, but it was still in good
working order.)

As of now, Hanshin Tramway is in its 23rd year of operation. It played
an important role as one of Hanshin's arteries.

Hankai Tramway is still popular with working class people. However, it
no longer enjoys status as a major artery connecting Osaka and Sakai.

\emph{Some railways were left untouched by the war.}

In addition to the Arima Railway, there were several other mountain
railroads I considered. The Mt. Atago railway was not included because
that was outside the game board's scope of central Keihanshin. Nose
Electric Railway was another option which was omitted in the end.

I wanted to include newly constructed railways such as the Kita Osaka
Kyuko Railway, Semboku Rapid Railway and Kobe Rapid Transit Railway.

I wanted to pick up the newly constructed railway. Kita Osaka Express,
Senboku Expressway, Kobe Express, etc.

I wanted to build these after the main company was established.  1829
has a system like this, which I emulated.  In 1829, it was a steamship
company that put a token on a port town to receive \$ 40 each turn.
I applied this concept to the Semboku Rapid Railway.

The Kobe Rapid Transit Railway is the most unique railroad, in that it
does not own a train, but instead works to connect routes of other
companies.  This year, the Hanshin / Hankyu / Sanyo mutual service
finally completed a direct express connection between Umeda and
Himeji. Therefore, in the game, we have created a station token that
is shared among several railway companies.

Kita-Osaka Express is also active as an extension of the subway's
Midosuji Line.  This railway is now famous as the cheapest route in
the country.  In addition, at the time of the 1970 World Expo, a
dedicated station was established to transport large numbers of
passengers.

I added some Kansai flavor at the end. Of course, the aforementioned
formation of Kinki and the Kobe Rapid Transit Railway provide Kansai
theme, but as a Hanshin fan, I wanted to include the
Tigers. Certainly, if the Tigers win, the economic effect is
significant. (Well, they rarely win nowadays, but they were very
strong until the 1950s. I'm wondering what kind of economic effect
they had...)

The effects of the Hankyu Takarazuka and Land Commercial Code were
other items I wished to include. These make Hankyu a magnificent
company.

The Great Hanshin Earthquake was also a recent memory, and I wanted to
create rules that reflected my lived experiences. (Maybe if an
outsider created these rules it would be frowned upon, but as a Kansai
resident, I definitely wanted this rule for posterity.)

\emph{Some rules were changed during playtesting}

First, the Great Hanshin Earthquake rule was made optional because it
unilaterally disadvantaged those who owned companies in
Hanshin. However, it is a rule I would like to revisit.

Next, Hanshin Tramway was changed to include 1 share of Hanshin. Due
to the advantages of Hankyu, the establishment of Hanshin was delayed
and the map north of Osaka was poorly balanced. Historically, Hanshin
started 15 years before Hankyu and was more developed than Hankyu do
the large population along its route. Additionally, even though it
became famous through Hankyu, the so-called Land Commercial Code,
which developed railway lines through residential land, was actually
initiated by Hanshin. Hankyu's headquarters changed from Umeda (Osaka
Kita) to Toyonaka.

The track connections in Osaka were also changed. In yellow, JR was
only able to go from Osaka Kita to Osaka Higashi. However if JR was
not yet started Hankyu and Hanshin could pass through the JR token
spot and place tokens in Osaka Higashi. On the other hand, Keihan and
Kinki have the opportunity to place tokens in Osaka North, killing the
runs until green. Since green tiles can not be placed until Osaka
Municipal Electric Railway closes, this results in a rough play.

There remains an issue one private company which has not been
resolved. I had believed that Uji Electric Railway was located in Uji,
but in fact it was a predecessor to Sanyo Electric Railway. The reason
that a company named ``Uji Electric Railway'' was located near Kobe
was that it was run by a power company from Uji.

I was in fact quite ignorant of Kansai Railway history when I began
designing, and studied it during the game development process. A good
idea, right?

Finally, I would like to thank Mr. Izumi and other members of the
Sakura Association who participated in playtesting from the beginning and
gave their valuable opinions.

{\itshape Footnote:

After these designer's notes were written (August 1998), we decided to use Kobe
Tramway instead of Uji Electric Railway. Also, since the game has
various minor corrections, I would like to make a revised note
including them if there is the opportunity. --Izumi}

\section{Producer's Notes}
{\large Izumi (泉)}

My relationship with 1890 started 3 years ago, when Takasaki-san
brought it to the Sakura Association, saying, ``look what I
have...''. He had what appeared to be a map of Kansai divided into
hexes, and a game called 1890 which used rules from the 18XX
series. At the time, I decided to play it at a later date.

At the time, I did not know what 18XX was, so I learned 1856 from
Mr. Nakao of Soldier (Kyoto's game circle). I started to feel
excitement in my heart when I began to considered all the possiblities
offered by the game. In this way, the Railway Promotion Committee
(Izumi/Kamiya) of the Sakura Association was established, and I played
1856 like a monkey. As I recall, somebody said, ``what ever happened
to 1890?'', and Mr. Takasaki brought it to us to play for the first time.

Unlike 1856, 1890 has minor companies and late private
companies. ``What a refreshing game'', I laughed. After many plays, I
realized that the merit of the late private companies is their stable
asset value, which is useful for a self-bankruptcy play. I also
discovered that development from the east to the south of Osaka would
be difficult without the minor companies.

After several plays, revising the rules, map, and tile manifest, etc.,
I thought, ``This would be a waste not to share with the world.''. So
I decided make a version for public distribution.

It took until the eve of August 1998 for the first distribution
version to be produced. I knew there were problems here and there but
I wanted to distribute it on a trial basis, and so gave it out free
of charge.

After that, development of the next version was suspended due to
computer issues, which were not resolved until 1999. In March, the 3rd
edition was created. In August, an edition was created for the Summer
Comiket fair. In April 2004, version 3.2 was released. This included a
complete revision of the manual, but it was so messy that new mistakes
were introduced. This edition is the 4th edition with English language
support (except for the manual). Basically, if you remove all the bugs
from this version, 1890 will be finished, although the manual will
also require some touch ups.

Behind the development of 1890 was a year-long struggle concerning my
employment at Orient Computer Instruments Ltd., where I experienced
severe labor practices on repeated instances. I was asked to submit a
pledge not to join a labor union, but refused to do so. Upon joining a
union, I was immediately dismissed. Following a lawsuit, I was able to
rejoin the company, but was dismissed once more after enduring various
harrassments: changing workplaces, isolated workplaces, and having to
perform light packing jobs in front of the company president. Again,
the issue was settled in court, but this time I did not return to
work.  Instead I received settlement money, which is a result I was
satisfied with.

In April 2001, I enrolled in acupuncture college and decided to begin
a career path in traditional medicine. Although I suspect my days
ahead will be busy, I hope to find time to play games such as 18xx.
\end{multicols}
%%% Local Variables:
%%% mode: latex
%%% TeX-master: "1890rules-en-translation"
%%% End: