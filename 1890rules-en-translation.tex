\documentclass{article}

\usepackage{CJK}
\usepackage{amssymb}
\usepackage{enumitem}
\usepackage[utf8]{inputenc}
\usepackage{parskip}
\usepackage{hyperref}
\usepackage[table]{xcolor}
\usepackage{multirow}
\usepackage{multicol}
\usepackage[top=1in, bottom=1.25in, left=1.25in, right=1.25in]{geometry}
\usepackage{mathtools}

\title{1890}
\setcounter{secnumdepth}{4}
\begin{document}
\begin{CJK}{UTF8}{min}
\begin{multicols}{2}
\section*{Introduction}

\subsubsection*{1890 is a game about the railways of Kansai}
1890 is a member of the 18XX genre of railway development games that
is focused on the the development of railroads in the Keihanshin
region of Japan. Players will operate the JR, Hankyu, Hanshin, Keihan,
Kinki, Nankai, Sanyo, and Osaka Municpal Subway major railway
companies, along with several smaller companies to achieve
victory. The game simulates the period of time from the beginning of
the railway era in Keihanshin to the year 2000, although some
scenarios may end earlier.

\subsubsection*{1890 is a game about asset growth}
1890 simulates stock investing and railway company management. At the end
of the game, the player with the greatest net worth will be the
winner. This is calculated as the combination of cash on hand and
the value of shares held.

\subsubsection*{1890 is a game about railway company management}
1890 has several types of railway companies, each with unique
features. Private companies and late income companies pay a fixed
income. Minor companies operate small trains. Public companies have a
stock price which fluctuates depending on operating conditions, and
can lead to greater profits. If you become president of a railway
company, successful management will be to your own benefit.

\subsubsection*{1890 is a game about stock trading}
Naturally, holding many shares of a well-performing company can lead
to victory, but companies will not always perform well. As time goes
on, trains become less plentiful and high-value trains can be traded
cheaply between companies. Selling shares in a company that is (or
expected to be) performing poorly can place a burden on the other
shareholders. Beware of other players' operating strategies and stock
speculation!

\subsubsection*{1890 is a multiplayer game}
It is difficult to be profitable alone. Aside from the initial seating
order, there are no random components in the game. Everything else is
in the hands of the players.

\section{Play of 1890}
This game can be played with 2 to 7 people. The target age is 12 years
and up, but 18 and older is recommended. There are a number of
scenarios and variants.

\subsection{Game Rounds}
1890 is played over a series of game rounds. A game round consists of
1 stock round followed by 1 to 3 operating rounds. These rounds are
tracked next to the stock market board. The number of game rounds is
determined by the scenario.

\subsection{Stock Rounds}
During the stock round, each player can choose to sell
shares, buy, sell then buy, or pass. Company shares are the unique
means for a player to increase their net worth.

Until all private and minor companies are purchased, an initial stock
round will be played, with different procedures than normal stock
rounds. Once the initial stock round is over, the game will continue with
the regular stock round.

\subsection{Operating Rounds}
During the operating round the president of each railway company
operates the company. First, all private companies operate, followed
by late private companies, minor companies, and finally public
companies. Minor companies are operated in order according to their
number, and public companies are operated in order of stock
prices. The president of each company makes all decisions for that
company during its turn.

When there are multiple operating rounds, move the round marker to the
corresponding position of that operating round. Unless the game ends during
the operating round, a new game turn will begin afterwards.


\subsection{Game Phases}
The game of 1890 progresses over 6 phases. The game phase is updated
as new types of trains are purchased. When entering a new phase, some
rules may change: the number of operating rounds, types of track tiles
which may be laid, train limits, etc. vary over the course of the
game.


\subsection{Game End}
The game ends when the bank is exhausted, any player goes bankrupt,
or the last game turn is over.


\subsection{Scenarios [A1-]}
The 'Hanshin Railway' and 'Keihan Railway' scenarios are introductory
scenarios which finish in less time. The full '1890' scenario finishes
in 3-5 hours. Please decide the scenario and variant before playing.


\section{Game Etiquette}
Each player plays so that everyone can enjoy the game. Play will be
smoothed by players planning their stock trades or company operations
ahead of their turn.


\subsection{Disclosure of player share and cash holdings}
\label{player-cash}
Cash and share holdings of each player are public knowledge. Players
should arrange their possessions such that they are easily seen by
others, and must respond truthfully to questions. However, excessive
inquiries will slow the game and should be avoided.


\subsection{Company Charters}
\label{company-charters}
The president of each railway company places trains and private
companies owned by the railway company on top of the company sheet so
that everyone can see it. Company funds are also placed on the company
sheet, but the amount does not need to be publicized.


\subsection{Confusion of Assets}
Players must take care to separate their assets from those of the
companies that they manage.


\subsection{Solo Play}
When a beginner player mixes with a enthusiast player, it is foolish
that the enthusiast player should go ahead with the game so that the
 beginner player can enjoy the game. To inductive players, is adopted
other specific play, let's put it is often not fun even have
participated each other care. And as many as one 18XX player.

\section{Components}
The 1890 is distributed as a kit. For kit production, please follow
the procedure in the attached sheet.

\subsection{Manual}
This booklet, which contains 1890 rules

\subsection{Map}
The map of 1890 is centered around Osaka and is divided into
hexes. A track tile may be placed in each hex, and each
railway company will develop a line for operation.

\begin{description}
\item[Green Areas] Track tiles may not be placed in the Green area
  that is not divided into hexes. Placing track tiles such
  that they connect to this area is prohibited.
\item[Sea Areas] The blue areas not divided into hexes are the
  sea. Track tiles can not be placed here. Placing track
  tiles such that they connect to this area is prohibited.
\item[Red Areas] Red areas represent off-board destinations. Track
  tiles may not be laid here, but track may connect to the black
  arrows. These areas are treated the same as a city when running a
  train.
\item[Map Area] An area divided into hexes where the track
  tiles can be placed.
\item[Plains] White hexes are plains. Tiles that do not include cities
  can be placed and upgraded here.
\item[Large Cities] Large city tiles may be placed and upgraded
  here. Placement and replacement of some track tiles can incur costs.
\item[Small Cities] Only yellow small city tiles can be placed.
\item[Double Small Cities] The Daito and Shijonowate hex is the only
  double small cities hex on the map. A terrain cost must be paid to
  lay a track tile here.
\item[Rivers] Rivers are preprinted in blue on the map. The river
  section is flanked by hexes. It is forbidden to place track tiles
  connecting to a river hexside. In addition, there are many rivers flowing
  through hexes that have a cost to lay track tiles.
\item[Mountains] Brown triangles indicate mountain hexes. Tiles
  without cities may be laid and upgraded, but a terrain cost must be
  paid for the initial tile lay.
\item[Grey hexes] No tiles may be laid or upgraded in grey hexes.
\item[Yellow hexes] Green track tile may be upgraded here
\item[Yellow hexes with printed track] Green tiles may be upgraded
  here. Because there are many special tiles, please be careful when
  replacing track tiles.
\end{description}


\subsection{Share Certificates}
Shares for private companies, late private companies, minor companies, and
public companies.


\subsection{Public company and Minor Company Charters}
This sheet is used when operating public and minor companies.  These
charters are managed by the president of the company. Company funds
and trains, unused station tokens, and purchased private companies are
placed on top of the charter.


\subsection{Track Tiles}
The map will be developed by laying and upgrading track tiles. There
are yellow tiles, green tiles, and brown tiles, and there are special
tiles that can only be placed in specific places. [Reference: Track tile
table (end of volume)]


\subsection{Money}
Paper banknotes in denominations of 1円, 4円, 5円, 10円, 20円, 35円,
50円 are (24 notes each), 100円 (30 notes), and 500円 (12 notes). An
additional 12 notes each of 500円 and 1000円, are available for use
after the bank breaks. The number of available banknotes may be
reduced depending on the scenario.

\subsection{Priority Deal Card}
This card indicates turn order in the stock round.


\subsection{Train Cards}
Shows a train

\subsection{Station Tokens}
Markers used to operate each public company. The trains that a company
runs must pass through a city where its station tokens are located. A
public company's unused stations tokens are placed on the company
charter. Minor companies' order tokens are used as their station
token.


\subsection{Order Tokens}
Used to determine turn order at the beginning of the game. After
that, they will be used as station tokens for minor companies or the
Kobe Shinkansen Railway late private company.



\subsection{Stock price tokens}
The square token containing the company name is used to indicate the
stock price of the public company. The one with the front and back is
used in the stock price table, and the one with only one side is used
in the par price table.


\subsection{Stock Market Board}
The stock market board is used to manage game turns, rounds, and stock
prices for public companies. Place each marker or token in its
designated space and be careful not to shift the position
inadvertently.


\subsubsection{Game Turn Tracker}
The current game turn is indicated by the game turn marker. Every time
the game turn is changed, the position of the marker is
updated. Depending on the variant, the game ends after a predefined
number of turns.


\subsubsection{Round Progress Marker}
The current round is represented by a round marker. It will be
updated each time a new stock round or operating round begins.


\subsubsection{Open Market}
Sold shares and discarded trains are placed in the open market.


\subsubsection{Par Value Table}
The par value table determines the price of a public company's
unsold stock. This price does not change. When the president's share
of a public company is purchased, the company's par value token is
placed within the par value area.


\subsubsection{Stock Price Table}
The stock price table tracks the changing stock prices of public
companies. The stock price of a public company is indicated by the
stock price token on the stock price table, and moves up and down by
buying and selling stocks, and moves left and right when dividends are
paid or withheld. Which section a public company's stock is in has an
important meaning in the game. When selling a public company stock
certificate or purchasing a stock certificate from the open market, it
is traded at this price. Also, the stock certificates held by the
players at the end of the game will be converted to the prices on this
stock table.


\paragraph{Red framed area}
Denotes the values which may be chosen as the initial stock price for
newly started companies. This area has no other effect on play.


\paragraph{White Section}
No special limitations are in play for a company whose stock price is
in this section.


\paragraph{Yellow Section}
Share certificates of companies whose stock price is in this area
are not counted towards the certificate limit.


\paragraph{Brown Section}
Share certificates of companies whose stock price in this area
are not counted toward the certificate limit. Also, you
may own shares of companies with stock price in this area beyond the
60\% limit.

\paragraph{Closing Section}
Companies whose stock price token has enters the closing
section will be closed immediately [Ref. 7.4 company closing].


\section{Game Preparation}
You need 3m\textsuperscript{2} or more space to play 1890. In order to
be able to see the player's assets etc., it is recommended to play on
a large table. The number of game components used will vary depending
on the scenario.


\subsection{Game Setup}
Select the scenario to play and prepare the game as follows.
\begin{itemize}
\item Place the map and the stock market boards in the center of the table.

\item Share certificates for public companies should be placed on the
  map, with the president's share at the top. Do the same with the
  late private company certificates.

\item Place each train in the appropriate space on the map according to type.

\item Elect a banker and arrange bank cash to be visible to all players

\item Place track tiles, company charters, station tokens, par
  value tokens, and stock price tokens by the map and stock market
  boards. Place the game turn marker on the first space of the turn
  tracker. Place the round marker on the stock round space of the
  round tracker.

\item Reserve space in front of each player to place a shares, company
  charter, money, etc.

\item Arrange private and minor company certificates in the order
  indicated by the chosen scenario.

\end{itemize}


\subsection{Bank}
Choose one player to be the banker. The banker controls the inflow and
outflow of bank funds. Bankers need the ability to be careful and
manage the bank's money during play so that it does not get confused
with the assets belonging to themselves or their railway companies.
Players who are familiar with the game are generally the banker. 

\subsubsection{Start funds}
Before starting the game, the banker prepares the necessary banknotes
in the scenario to be played, and distributes the starting money to each player.

\subsubsection{Bank cash entry and exit}
Payments for dividends from each railway company will be made from the
banks. Payments for new trains, trains in the open market, terrain
costs, and stock certificate purchases will all be made to the bank.

\subsection{Order}
Each player draws an order token placed in a cup etc. one by one to
determine the order. The banker sits in a convenient place to handle
the money. Other players center their seats in a clockwise order, in
ascending token number, centered on the banker. There are two number 4
tokens, but the darker one is used as the number 7 token. The player
with the number 1 token receives the priority deal card.


\section{Game phases} \label{sec:game-phases}
The game phases represent the progress of time and carry signifcant
influence to the game. Each game phase changes represents a
breakthrough in railway development history.

The game phase is updated as soon as a new train type is purchased.
The purchase of the first 2+2 and 3+3-trains triggers the start of the
second half of phase 1 and 2 respectively. This may or may not have
gameplay effects depending on the scenario.

1890 consists of six game phases. More than one phase may occur in
one operating round. In addition, the game may end without arriving at
the sixth phase. The following is an explanation of the changes in the
rules that occur in each phase.

\paragraph*{Phase 1 (first half, second half)}
The first half of phase 1 is from the beginning of the game until
the purchase of the first 2+2-train. The second half of phase 1 is
until the purchase of the first 3-train. During this time, the
following restrictions apply:
\begin{itemize}
\item Only yellow track tiles can be placed.
\item JR West Japan can place or upgrade two tiles.
\item The train limit for minors companies, public companies and the JR is
  2, 4 and 6 respectively.
\item Public companies can not purchase private companies.
\item 1 operating round per game turn
\item Red offboard spaces are valued using the leftmost number.
\end{itemize}

\paragraph*{Phase 2 (first half, second half)}
The first half of phase 2 is from the purchase of the first 3-train
until the purchase of the first 3+3-train, and second half of phase 2
is until the purchase of the first 4-train. During this time, the
following restrictions apply:
\begin{itemize}
\item Yellow and green track tiles can be placed.
\item JR West Japan place or upgrade two tiles.
\item The train limit for minors companies, public companies and the JR is
  2, 4 and 6 respectively.
\item Public companies can purchase private companies.
\item Two operating rounds per game turn (starting on the game turn after
  the first 3-train is purchased).
\item Red offboard spaces are valued using the leftmost number.
\end{itemize}

\paragraph*{Phase 3}
Phase 3 begins with the purchase of first 4-train until the
purchase of the first 5-train. The following restrictions apply during
this phase:
\begin{itemize}
\item Yellow and green track tiles can be placed.
\item JR West Japan can place or upgrade two tiles.
\item The train limit for minor companies, public companies, and JR is
  1, 3, and 4 respectively.
\item Public companies can purchase private companies.
\item 2 operating rounds per game turn.
\item All 2-trains are rusted.
\item Red offboard spaces are valued using the leftmost number.
\end{itemize}

\paragraph*{Phase 4}
Phase 4 begins with the purchase of first 5-train until the
purchase of the first 6-train. The following restrictions apply during
this phase:
\begin{itemize}
\item Yellow, green and brown track tiles can be placed.
\item JR West Japan can only place or upgrade 1 tile.
\item The train limit for minor companies, public companies, and JR is
  1, 2, and 3 respectively.
\item All private companies are closed. (Exception: Hankan Electric Railway)
\item From the next game turn, 3 operating rounds per game turn.
\item All 2+2-trains are rusted.
\item Red offboard spaces are valued using the middle number.
\end{itemize}

\paragraph*{5th phase}
Phase 5 begins with the purchase of first 6-train until the
purchase of the first D-train. The following restrictions apply during
this phase:
\begin{itemize}
\item All track tiles can be placed.
\item The train limit for minor companies, public companies, and JR is
1, 2, and 3 respectively.
\item D-trains can be purchased
\item All 3-trains and 3+3-trains are rusted.
\item 3 operating rounds per game turn.
\item Red offboard spaces are valued using the middle number.
\end{itemize}

\paragraph*{6th phase}
Phase 6 begins with the purchase of the first D-train and continues
to the end of the game. The following restrictions apply during this
phase:
\begin{itemize}
\item All track tiles can be placed.
\item The train limit for minor companies, public companies, and JR is
1, 2, and 3 respectively.
\item All 4-trains are rusted.
\item 3 operating rounds per game turn.
\item Red offboard spaces are valued using the rightmost number.
\end{itemize}


\section{Shares}

\subsection{Shares and certificates}
The 1890 uses stock certificates to represent the ownership of each railway
company.

A private company, minor company, or late private company's single
certificate represents 100\% ownership of that company. One share
certificate of a public company represents either 10\% or 20\% of ownership,
and each counts as one certificate towards the certificate limit.

When each railway company pays dividends during the operating round,
the profit is distributed to each shareholder according to this
percentage.

\subsection{Certificate Limit}
There is a limit to the number of certificates that each player can
own (private companies, minor companies, late private companies, and
public company). The certificate limit is specified for each scenario
and depends on the number of players.

\subsubsection{Exception to the certificate limit}
\label{sec:certificate-limit-exceptions}
If the stock price of a public company is in the yellow or brown
section of the stock price table, its stock certificates do not count
towards the certificate limit.

\subsection{Ownership Limit}
Players can usually only own up to 60\% of a public company's
shares.

\subsubsection{Exception to the ownership limit}
\label{sec:ownership-limit-exceptions}
If a stock company's stock price is in the brown section of the stock
price table, players can hold shares of that company beyond the 60\%
ownership limit.

\subsection{Exceeding certificate and ownership limits}
If the stock token moves from the yellow or brown section during the
operating round or at the end of the stock round, this may cause the
certificate and/or ownership limits to be exceeded. In this case,
affected players must sell shares in their next stock round turn to
bring them in compliance with the limits.

\subsection{Change of president} \label{shares-change-president}
A change of president may occur when a player purchases a stock
certificate or a president sells their stock certificates.

If a public company has a player who owns a percentage of shares of
the company more than the current president, that player will become
the new president and will have the operating authority and all
responsibility for the railway company. When two or more players have
the same number, the nearest player clockwise from the leaving
president becomes the new president. The new president will exchange
two of their regular stock certificates for the president's
certificate, and receive the company charter and all its holdings
(trains, station tokens, private companies, funds).


\section{Starting and closing of companies}\label{sec:starting-companies}
Private companies, minor companies, and late private companies are
started at the time of purchase by a player. Public companies are
established when a fixed number of shares are purchased from the
initial offer.

\subsection{Starting private companies and late private companies}
Private companies and a late private companies are started at the
moment their certificate is purchased. The purchasing player is the
president.

\subsection{Starting minor companies}
Minor companies are started when their certificate is
purchased. The player who has purchased the certificate will be the
president. The minor company certificate is also used as its company
charter. Once the company is started, the
president receives capital from the bank and places it on the company
charter and the corresponding the order number token is placed as its
station token.

\subsection{Starting public companies}
A public company is started when 5 shares (50\%) are purchased from
the initial offer. Exception: In the 1890 scenario, Kinki Nippon
Railways is started when the owner of the Osaka Electric Railway minor
company announces the start of Kinki Nippon Railways.

\subsubsection{Purchasing the president certificate of a public company}
\label{floating-par-price}
Usually, the first stock certificate purchased from a public company
is the president's stock certificate. The player who buys the
president's stock certificate determines the par price of the public
company from the following options: 65円, 70円, 75円, 80円, 90円, or
100円. They place the par token in the par price table, and place
the stock price token in the red-outlined are in the stock price
table. If other companies' tokens occupy that space, the new company's
token is placed at the bottom of the stack. The price to purchase the
president's certificate is two times the par price.

\subsubsection{Preparation for starting a public company}
\label{floating-initial-capital}
The president of the started public company receives the company
charter and its station tokens, placing them on the spaces on the
company charter marked with prices 40円 and 100円. As capital,
the company receives ten times the par price from the bank. For the
remainder of the game, further sale and purchase of the company's
shares will not affect the amount of funds in its treaury. [Exception:
Kinki, \autoref{kinki}]

\subsection{Company closure}
Private companies will be closed at the beginning of Phase 4
(exceptions: Hankan Electric Railway, C8.4, D8.5: Osaka City Tramway).

A public company closes when its stock price token enters the closed
space on the stock price table. The closed company's stock
certificates are discarded. The treasury of the public company are
returned to the bank, and its trains are discarded to the open
market. All station tokens placed on the map will also be removed,
vacating their city spaces and allowing other railway companies to
place station tokens there. Closed public companies are no longer
available to start for the remainder of the game.


\section{Stock round}
In the stock round, players buy and sell railway company shares, which
form the source of players' income during the game. While private and
minor companies remain unsold, a special initial stock round takes
place instead of a regular stock round.

As the presidency of public companies is decided by which player has
the largest share holdings, buying and selling of shares is important.

\subsection{Priority Deal Card}
Beginning with the player holding the priority deal card, players
perform turn actions in clockwise order.

If the player holding the priority deal card performs an action
instead of passing, they will pass the card to the next player in
clockwise order. This is true for both regular stock rounds and
initial stock rounds.

\subsection{Progress of the stock round}
The stock round starts from the player with the priority deal card,
and proceeds by each player taking turns in clockwise order. On their
turn, a player may perform one of the following four actions:
\begin{itemize}
\item Sell shares
\item Buy 1 stock certificate.
\item Sell and buy shares (in either order)
\item Pass
\end{itemize}

A player who takes an action other than passing will give the priority
deal card to the next player in turn order.

The stock round will continue to be played until all players pass in
succession, and may go around the table several times. Players who
pass may take future actions in the stock round, provided the stock
round has not ended before the turn returns to them.

\subsection{Selling shares}
Only shares of public companies may be sold. During another player's
turn, it is possible to sell private companies if the other player
requests to purchase them.

\subsubsection{Fall in stock price due to the sale of shares}
\label{sr-sell-price-drop}
When a player sells a stock certificate, they place the stock
certificate in the open market and receive the current value according
to the stock price table. The railway company's stock price token is then moved
down one row for every 10\% of the shares sold. When the stock price token
is on the bottom row, the stock price does not change, regardless of
how many shares are sold.

\subsubsection{Restrictions on share sales}
\label{sr-sell-restrictions}
No more than 50\% of the shares of any railway company may be in the
open market. Stock sales that would exceed this limit are not
allowed.

The president's certificate can not be sold unless another player
has at least two shares of the company.

During the stock round of the first game turn, all players are
prohibited from selling stock.

\subsection{Buying shares} \label{sr-buy}
Players can purchase one stock certificate from the initial offer, or
one stock certificate from the open market. Players may also buy one
private company from another player.  Stock certificates from the
initial offer will be purchased in order, beginning with the
president's certificate, at the par price decided by the first player
to buy shares of that company. Stock certificates in the open market
are purchased at the price currently indicated on the stock price
table. The president's certificate and four regular certificates of
Kinki Nippon Railway (see 1890 scenario) appear as minor companies
convert. The president's share of Osaka Municipal Subway is given to the
purchaser of the Osaka Municipal Electrica Railway private company.

\subsubsection{Restrictions on share purchases}
Players may not buy shares of a company which they have sold
previously in the same stock round. In addition a player may not
purchase shares which would cause them to exceed certificate or
ownership limits. (Exception: \autoref{sec:certificate-limit-exceptions},
\autoref{sec:ownership-limit-exceptions})

During the first game turn and the initial stock round, late private
companies may not be purchased.

The purchase of certificates from other players which are not private
companies is prohibited (see: 10.9)

\subsection{Sell and Buy Shares} \label{sr-sell-buy}
Players may choose whether to sell or buy first, but all sales must be
done at once. If a company's stock price is higher than its par price,
it is possible to buy one certificate from the initial offer and
immediately sell it for a profit.

\subsection{End of stock round} \label{sr-end-of-sr}
The stock round ends when all players pass in succession. If no
player performed actions, the priority deal card does not move. When
the stock round is over, move the round marker from stock round to
operating round 1.

\subsubsection{Price adjustment for sold out companies}
\label{sr-sold-out}
At the end of the stock round, stock price tokens for public companies
whose shares are all owned by players move up one
row. If the stock price token is already on the top row, it does not move.


\section{Initial stock round}\label{isr}
An initial stock round will be played until all private and minor
companies appearing in the chosen scenario have been purchased by
players.

\subsection{Progress of the Initial Stock Round}
Each private and minor company is arranged for sale in the order shown in the
scenario. The initial stock round is resolved clockwise from the
player with the priority deal card. Each player performs one
of the following in his turn, and passes the priority deal
card to the next player in clockwise order.

\begin{itemize}
\item Buy a purchaseable company.
\item Bid for one of the auctionable companies.
\item Pass.
\end{itemize}

\subsubsection{Available Railway Companies and Auctioned Companies}
The only railway companies that players can buy are the companies at
the very beginning of the list of companies. The companies listed
after that will are auctionable railway companies.

\subsubsection{Purchase Available Railway Company}
The active player may buy the purchaseable company at face value, then
passes the priority deal card to the next player.

If the next company in line has no bids on it, it will become the new
purchaseable company. Otherwise, if one or more players have bid on
it, the auction will be carried out.

\subsubsection{Participate in the auction of a competitive railway company}
The active player may place a bid on one of the auctionable railway
companies and then pass the priority deal card to the next
player. Over the course of multiple turns, a player may have bids on
several railway companies, if they have enough money.

\paragraph{How to bid}\label{isr-bidding}

The initial bid on a railway company must be greater than its face
value by 5円 or more. Subsequent bids on that company must exceed
the highest bid by 5円 or more. The player who makes a bid must set
aside money equal to the value of the bid. This money is unavailable
for other use until the company they bid on is sold. It is possible to
increase the value of a previously placed bid by adding more money to
these reserved funds, but there is no requirement to do so. After
bidding, pass the priority deal card to the next player.

\subsubsection{Auction}
When a purchaseable railway company is bought, and the next company in
line has bids on it, an auction will be resolved for that
company. Only players who have placed bids on the company will
participate in the auction. If only one player has bid, the company
will be sold to them at their bid value. If multiple players have
bids, then the auction will commence clockwise from the player with
the highest bid.

To remain in the auction, players must increase their bid to exceed
the current high bid by 5円 or more. Otherwise, they pass and are
removed from the auction. When all but one player have passed, the
remaining player purchases the railway company for their bid value.

After resolving the auction, if the next company in line also has bids
on it, resolve a new auction for that company. Otherwise, the next
company becomes the new purchaseable company and the initial stock
round continues.

Regardless of how many companies were auctioned and to whom, the
priority deal card is passed to the player clockwise from the player
who made the purchase that triggered the auction(s).

\subsection{End of the Initial Stock Round}
The initial stock round ends when all private companies and minor
companies have been purchased by players. After the initial stock
round ends, a regular stock round is played. During the initial stock
round, late private company and public company certificates can not be
bought. In addition, late private companies can not be bought on the
first game turn.

\subsection{Initial stock round across multiple game turns}
The initial stock round may be played for two or more game turns. If
all players pass consecutively before all the private and minor
companies are sold, the stock round ends and an operating round is
played. Private and minor companies which have been sold will
operate. The game turn is incremented and the stock round of the next
game turn will continue from the previous intital stock round. Bids
from the previous game turn remain on the companies and their corresponding
funds must be kept in reserve.

\subsubsection{Arima Electric Railway purchase price decrease}
\label{isr-arima}
Each time one game turn ends without the Arima Electric Railway being
sold, its purchase price drops 5円. If this makes the purchase
price 0円, the player with the priority deal card must buy it in
the next initial stock round. This is considered as the player
having purchased Arima and any consequences that would entail from
such a purchase are carried out (e.g. passing priority deal, triggering
auctions). Arima Railway is the only railway company whose purchase
price falls in this manner, and this does not affect its face value.

\subsection{Determination of stock price (1890 scenario)}
The player who purchases the Osaka Municpal Electric Railway private
company immediately sets the par price for the Osaka Municipal Subway
public company. The player who purchases the Osaka Electric Railroad
minor company immediately sets the par price for the Kinki Nippon
Railways public company. (C8.1, C8.2, D8.1)



\section{Operating Round}
In the operating round, all railway companies that can operate will do
so in turn order. There will be 1 to 3 operating rounds per game turn
depending on the game phase.

\subsection{Progress of the operation round}
The operation round will proceed in the following steps.
\begin{enumerate}
\item Operate private companies and late private companies.
\item Operate minor companies in numeric order (1-5).
\item Operate public companies in order of descending stock price.
\end{enumerate}

\subsection{Operating private companies and late private company}
All private companies and late private companies operate
simultaneously.  The dividend amount printed on their certificate is
paid to their owner.

\subsection{Operating minor companies and public companies}
\label{or-operating-order}
First, minor companies operate in numeric order. Next, public
companies operate in the order of descending stock price. If two
public companies have the same stock price, the company whose stock
price token is further right will operate first. If the two companies'
stock price tokens are in the same space, the company at the top of
the stack will operate first.

A minor or public company's operating round turn consists of the following steps:

\begin{enumerate}
\item Lay and upgrade track tiles. (Optional)
\item Place a new station token. (Optional, public company only)
\item Run trains for revenue.
\item Public companies pay dividends or withhold. Minor companies will
  pay dividends.
\item Move the stock price token in the stock price table. (Public company only)
\item Buy trains. (Optional)
\end{enumerate}

The above actions must be performed in the given order, although not
all actions must be performed. Additionally, public companies may
purchase private companies from players at any point during their
operating round turn, if permitted by the current game phase.

\subsubsection{Track tiles}
The railway company can construct the track by placing or upgrading
the track tiles on the map. This action will connect the city to the
train so that it can travel and increase the value of the city through
urban development. Track tiles are placed to fit into any hex on the
map. Track tiles once placed become part of the map and can only be
moved when upgraded by other track tiles. The placement and upgrading
of the track tiles is optional. Also, companies do not need to use the
track tiles they placed when operating their trains.

\paragraph{Laying and upgrading track tiles} \label{or-laying-tiles}
Each railway company can place or upgrade one track tile, except for
JR, which may place or upgrade up to two in phases 1 to 3. Tiles can
be placed or upgraded in hexes that are connected to one of the
company's station tokens, regardless of distance.

Yellow tiles are laid on empty map hexes, and are then upgraded to
green and brown. Brown track tiles may not be laid directly on the
map, and yellow tiles may not be upgraded directly to brown.

Tile upgrades are not permitted until phase 2.

\paragraph{Restrictions on track tile laying}
\label{track-lay-restrictions}
Track tiles may not be laid such that they connect to non-hex
locations, Osaka Bay, non-tracked edges of red offboard hexes, heavy
black borders, and hexes not included in the current scenario.

Track tiles may also not be placed or upgraded in grey hexes
(Tanigami, Uji, and northeast of Nara).

Certain hexes are blocked by private company track. No track tiles may
be laid in these hexes until either: the corresponding private company
is bought by a public company, closes, or otherwise loses placement
restrictions.

\paragraph{Laying track tiles}
Yellow track tiles are laid on empty hexes without yellow or black
borders on the map. Only yellow track tiles with large cities may be
placed on empty hexes with large cities. Only yellow track tiles with
the same number of small cities can be placed on empty hexes with a
small cities. Track tiles placed in a small city can not be further
upgraded for the rest of the game.

\paragraph{Upgrading track tiles}
Track tile may be upgraded in any direction that preserves all track
and connections on the previous tile, but upgrades which would violate
track lay restrictions (\autoref{track-lay-restrictions}) are
prohibited. The previous track tile is removed from the map and is
available for use elsewhere.

When upgrading a track tile, existing station tokens must be in the
same position on the new track tile.


\paragraph{Costs to lay or upgrade track tiles}
Certain hexes on the map are printed with a terrain cost that must be paid in
order to lay or upgrade a track tile there. This cost must be paid
from the treasury of the operating railway company, and the president
can not contribute personal funds in the case of a shortfall.

\paragraph{Special track tiles}
Large city hexes with yellow track tiles printed on the map (Osaka Kita,
Osaka Higashi, Osaka Nishi, Osaka Minami, Kobe, Kyoto, Nara, Hirakata,
Kashiwara) are treated as pre-laid yellow tiles.

Yellow-framed hexes on the map (Ibaraki, Moriguchi, Sakai) can be
upgraded with the green track tiles marked ``OO'' (\#210, \#211). These
are further upgraded with the brown ``OO'' track tile (\#217), combines
the two separate large cities into a single large city.

Amagasaki, Kobe, Kyoto, Osaka Higashi, Osaka Kita, and Osaka Minami
may only be upgraded with the green and brown track tiles marked with
their name.

Normal yellow and green track tiles are used for Nishinomiya, but the
only brown tile which may be upgraded there is the one marked with
``Nishinomiya'' (\#465).

Osaka Nishi upgrades to the green \#12 track tile, and then to the
brown tile marked ``Osaka Nishi'' (\#839).

\subsection{Station token placement}
Station tokens represent locations and facilities that a railway
company uses to operate its trains. Once station tokens are placed,
they can not be removed until the end of the game, or the company
closes. (Exceptions: Kinki Nippon Railways, Kobe Electric Railway).

\subsubsection{Home station token placement}
\label{or-home-station}
During a railway company's first operating round turn of the game, a
free station token is placed in its home location(s), indicated by the
company's logo printed in large city spaces on the map. (Exception:
Kinki Nippon Railways \ref{kinki})

\subsubsection{Normal station token placement} \label{or-normal-stations}
Each public company can place one station during its turn. The cost
for each token is listed on the company charter: 40円 for the first
token, and 100円 for each subsequent token. This cost is paid from
the treasury of the company placing the token. The number of station tokens
a company may own is limited by the number of spaces on the charter.

On a company's first operating turn, it may place a normal station
token in addition to its home token(s).

\subsubsection{Requirements for placing normal station tokens}
\label{or-station-placement-requirements}
Station tokens are placed in empty large city locations. The city must
be connected to one of the company's existing tokens on the map such
that a train of infinite length could connect the two
locations. Station tokens may not be stacked, and a company may only
have one of its station tokens on any single track tile.

Home station locations for unstarted companies are reserved and no
station tokens may be placed there. If a track tile upgrade adds
additional large city spaces, then other public companies may place
their tokens in the additional space.

\subsection{Operating trains}
The active company determines the routes that its trains will run. The
revenue generated is then distributed to the company's shareholders
and/or its treasury.

\subsubsection{Right to decide train routes} \label{or-route-choice}
The president of the operating company decides the routes that will be
run by company's trains. The company's other shareholders can describe
better routes to the president, but the president is not obligated to
follow their suggestions.

\subsubsection{Route definition} \label{or-route-definition}
Train routes must follow the track on the map according to the
following rules. If a railway company owns multiple trains, it is
possible for it to operate multiple routes.
\begin{itemize}
\item The train's route must include at least one city containing the
  company's station token and one other city (large city, small city,
  red off-board city).

\item The number of cities that the train can visit is the same as the
  train number (\autoref{sec:train-types} types of trains)

\item Trains can only travel in the direction of a straight or curved
  track. The track lines crossing each other on a tile are overpasses,
  so it is not possible to switch directions at such
  points. Backtracking is not permitted.

\item If a route contains a red off-board location, it must be the start
  or end of the route.

\item The same city can not be used twice on a single train's
  route. This includes red off-board cities. However, different cities on one
  track tile can be used.

\item Track may not be re-used in a single train route, but a
  track tile can be revisited if doing so uses only yet unused track.

\item A route may include cities where all spaces are occupied by
  station tokens, but the route may pass through such cities only if
  one of the tokens belongs to the operating company. (Exception: Kobe
  Rapid Transit Railway \ref{kobe})

\item Count all cities on the train route.

\item If operating multiple trains, a single city may be visited by
  more than one train as long as different track is used.
\end{itemize}

\subsubsection{Revenue from Operations}
The sum of the values of the cities (large cities, small cities, and
red off-board cities) on a train's route is the revenue for that
train. Total the revenue from each train to obtain the revenue for the
current operating round. Companies with no trains earn no revenue.

\subsection{Deciding whether to pay dividends or not}
A public company that operated trains may either pay the revenue as
dividends or keep it in its treasury. Minor companies that operated
trains must pay dividends. Companies that did not operate trains do
not pay dividends.

\subsubsection{Minor company dividends} \label{or-minor-dividends}
Minor companies can only pay dividends. Half of the revenue is paid to
the president and the other half is added to the company treasury.

\subsubsection{Public company dividends} \label{or-public-dividends}
If the president of a public company chooses to pay a dividend, the
revenue generated by the operating trains will be distributed among
shareholders. The company's shareholders receive 10\% of the revenue
for each share certificate (20\% for the president's
certificates). Shares in the open market pay to the company
treasury. Shares in the initial offer pay their dividends to the
bank. (Exception: JR \autoref{JR})

\subsubsection{No distribution of public companies} \label{or-withholding}
If the president of a public company elects not to give a dividend,
all the revenue generated by operating trains will be added to the
company's funds.

\subsubsection{Moving stock price tokens} \label{or-moving-stock-prices}
The stock price of a public company moves depending on whether it paid
dividends or kept the revenue for its treasury. If a dividend was
paid, move the stock price token to the right and flip it over. If it
did not pay a dividend, move the stock price token to the left and
flip it over.

\subsection{Buy a train}
Minor companies and public companies can buy trains as long as
the train limit is not exceeded. Check for phase changes in between
each train purchased (\autoref{sec:buying-trains}).

\subsection{End of operating round}
The operation round ends when all the operational companies have been
operated. Flip over all stock price tokens. If any started companies
are in a stack, their tokens will always be on the top.

Move the round marker of the round progress table to the next space;
if the current run round is the last operating round, move the round marker
to the stock round space and advance the game turn marker by one.

\subsection{Purchase of a Private Company in a Public Company}
\label{or-private-purchase}
During the phases 2 and 3, a public company can purchase a private
company from any player during its own operating round turn.  The
purchase price must be at least half and at most double the private
company's face value. For example, the 20円 Arima Railway is sold
between 10円 and 40円.

Payments are made directly from the treasury of the public company to
the player who held the private company. Some private companies give
benefits to the public company that owns it. This benefit can be used
immediately after purchasing a private company. Private companies that
are owned by public companies pay their dividend directly to the
owning company's treasury.

Private companies, late private companies, and minor companies can
not purchase private companies.


\section{Railway companies}
In 1890, there are 6 private companies, 5 minor companies, 4 late private
companies, and 8 public companies. By operating these companies,
players can earn a profit. Depending on the scenario, some companies
may not be used or their rules may change.

\subsection{Private companies}
Private companies are small privately-owned companies that are
purchased by players in the initial stock round. Each private company
owned by the player counts as 1 certificate towards the certificate
limit. All private companies will pay dividends printed on their
certificate at the beginning of each turn each operating round. The
player or public company that owns the private company receives the
dividend.

\subsubsection{Private company restrictions on track tile placement}
Certain hexes on the map are marked with private company tracks. While
these private companies are owned by players, any placement of track
tiles on these hexes is prohibited. This restriction is lifted when
the private company is bought by a public company or the private
company closes.

\subsubsection{Purchase of private companies by public companies}
Beginning in phase 2, public companies can buy private companies
any time during their own operating turn. The purchase price is any
amount between half and twice the private company's face value agreed
upon by the seller and buyer.  A railway company that purchases a
private company may receive certain benefits.

\subsubsection{Purchase of private companies by players}
During their stock round turn, players can buy private companies from
other players at any agreed-upon price.

\subsubsection{Closure of private companies}
Private companies close at the beginning of phase 4. All benefits from
a private company cease when it is closed.

\subsection{Late private company}
Late private companies are almost always treated the same as private
companies. From the second game turn, all players will be able to
purchase late private companies during their turn in a regular stock
round (not an initial stock round). Each late private company owned by
the player counts as one certificate toward the certificate
limit. Late private companies may not be sold during the game. At the
beginng of each operating round, the player which owns the late
private company receives the dividend printed on its certificate. Late
private companies do not close during the game.

\subsection{Minor companies}
Minor companies are specialized private companies that operate in the
same way as public companies. Minor companies are purchased by players
in the initial stock round. Minor companies that have been purchased
by players can not be sold during the game. The starting capital of a
minor company fixed and will not change regardless of the purchase
price. Minor companies may not be bought or sold between public
companies or players.

\subsubsection{Operation of a minor company}
Minor companies operate after private companies and late private
companies. The number shown in the small company's charter will be its
operating order, and the token corresponding to the number will be
used as the station token for that small company.

A minor company is a larger company than a private company or a late
private company, and operates like a public company until a merger
with Kinki Nippon Railways or a transformation to a late private
company occurs.

Minor companies must pay dividends whenever they operate trains. A
minor company pays half of its earnings to the president of that
company and the rest to the company treasury. The stock price of a
minor company does not change from the par value.

\subsection{Public companies}
There are eight public companies in 1890: Osaka Subway, Hankyu,
Hanshin, Keihan, Kinki, Nankai, Sanyo, and JR. They are major railway
companies that generate the greatest profits in this game. Public
companies are owned by multiple players by purchasing stock
certificates in a stock round. The president of a public company is
the player with the president's stock certificate and they operate the
public company during the operating round. The president may change
through the buying and selling of stock certificates.

Some companies listed in the scenario may be started or operated
in special procedures.

\subsubsection{Operation of public companies}
The president of a public company takes all of the company's actions
during the operating round: laying or upgrading track tiles, placing
station tokens, operating trains for revenue, deciding whether to pay
dividends or withold, purchasing trains, etc.

\subsubsection{Starting of public company}
If 50\% of a company's shares is held among all players and the open
market, then the public company is started. (Exception: D8.4 Kinki
Nippon Railways)

Once started, the public company will continue to operate until the
end of the game, unless it is closed. Once closed, the public company
will not appear again during the game.


\section{Trains}
Trains can only be owned by minor companies and public
companies. Trains make money by traveling between cities during the
operating round. Trains owned by the company can not be voluntarily
discarded.

\subsection{Types of trains}
\label{sec:train-types}
The type of train represents the number of cities (large cities, small
cities, red areas outside the map) where the train can run. For
example, 4-trains can run to four cities. As an exception, 2+2-trains
can run to 2 large cities and 2 small cities, and 3+3-trains can run to 3
large cities and 3 small cities. D-trains can run to an unlimited
number of cities.

The number of trains available is as shown in the table below, and the
more cities a train can run to, the more expensive it becomes. The
number of type of trains that are available is determined by the
scenario.

\begin{tabular}{|l|l|l|l|}
\hline
Train & Count & Price & Era \\
\hline
2 & 9 & 80円 & 1910 \\
2+2 & 3 & 120円 & 1920 \\
3 & 5 & 180円 & 1930 \\
3+3 & 2 & 230円 & 1940 \\
4 & 4 & 300円 & 1950 \\
5 & 3 & 450円 & 1960 \\
6 & 2 & 630円 & 1970 \\
D & 6 & 1100円(800) & 1980 \\
\hline
\end{tabular}

\subsection{Train limit}

The maximum number of trains that a railway company can own is fixed
for each game phase. A railway company can not own or purchase a
train beyond the train limit.

If the train limit is reduced when the game phase changes through the
purchase of a train, the presidents of each company must immediately
discard any trains the company holds exceeding the limit. Discarded
trains are placed in the open market and are available for purchase.

\subsection{Train ownership obligation}
\label{train-obligation}
Companies are obligated to own a train if they can trace a valid route
from any of their station tokens. If such a company does not own a train at
the end of its turn, it must buy one. If the company treasury is
insufficient to do so, a forced-purchase occurs.

A company is exempt from owning a train if it has no route, or if
there are no trains available for purchase from the bank or open
market. However, in these cases, the railway company can not pay
dividends, so the stock price will continue to fall each operating
round.

\subsection{Purchasing trains} \label{sec:buying-trains}

All trains are initially purchased from the bank. Trains purchased
from the bank will be sold in order. For example, 2+2-trains cannot be
bought until all 2-trains are purchased from the bank, and 5-trains
cannot be bought until all 4-trains are purchased from the
bank. However, once the first 6-train is purchased, D-trains may be
bought. The purchase of a new train type will trigger a change of game
phase. The cost to purchase a train is paid from the treasury of the
railway company that purchases the train. (Exception: forced purchase
of trains 12.5)

\subsubsection{Purchase by the trade-in of the train}

The price of the train is listed on the train card. D-trains can be
purchased at 1,100円 or at 800円 by trading in a 4-, 5- or 6-
train. Since 4-trains will be rusted when the first D-train is
purchased, it is only the first D-train may be bought with the
trade-in of a 4-train. Traded-in trains are placed in the open market.

\subsubsection{Purchasing trains from the market and other companies}
\label{trains-buying-from-companies}

Trains can be purchased from other railway companies and the open
market.  A company may buy a train from a different company at any
price agreed upon by the two presidents, that is at least 1円. Trains
in the open market are bought at face value.

\subsection{Forced purchase of trains} \label{train-force-buy}
If a company is obligated to own a train, and does not have enough
money in its treasury to buy one from the bank or open market, the
president of the company must use their own funds to make up the
shortfall. The following rules apply for a forced purchase:

\begin{itemize}
\item Only one train may be bought.
\item The cheapest train from the open market or bank must be
  bought. Buying a train from another company is not allowed.
\item The company's entire treasury is spent, and the remaining amount
  is paid by the president.
\end{itemize}

If the president doesn't have enough money to pay for the shortfall,
they must sell stock to raise the necessary funds. (Refer to the
section on forced sale of share certificates).

\subsection{Train rusting}

2-, 2+2-, 3-, 3+3-, and 4-trains are rusted at the
beginning of game phase 3, 4, 5, and 6 respectively. Rusted trains are
immediately and unconditionally removed from the game
(\autoref{sec:game-phases}).


\section{End of the game}
The game may end in three ways. A player goes bankrupt, the bank runs
out of money, the end of the final game turn (according to the
scenario).

When the game is over, the player with the greatest assets will be the
winner. It is possible for a player who has gone bankrupt to
win. Count the following towards personal assets:

\begin{itemize}
\item On-hand cash held by players
\item Shares valued by their price on the stock price table
\item The face value of private companies, a minor companies, and late
  private companies
\end{itemize}

\subsection{End processing of the game}

In order to make the game termination process more error-free, we suggest
the following procedure:

\begin{enumerate}
\item Return all cash remaining in each company's treasury to the bank.
\item Exchange all stock certificates for cash. Make the exchange for
  all players at the same time, starting with the company with the
  lowest stock price.
\item Exchange private companies, minor companies, and late private companies
  for cash at face value
\item Compare all players' cash on hand.
\end{enumerate}

\subsection{Player Bankruptcy}
\label{endgame-player-bankruptcy}
If a player can not provide the funds needed to make a payment, that
player will go bankrupt. No player or company can do anything from
that moment on.

A player may not voluntarily declare bankruptcy.

If a minor company or a public company is obliged to purchase trains,
and company treasury plus president's cash after all possible share
sales is insufficient to buy a train, the president player is
bankrupt. All money raised is seized by the bank and the remaining
debt is waived. The game immediately ends.

\subsection{Breaking the bank}
\label{endgame-bank-break}
The bank breaks when there is insufficient money to pay players or
companies, e.g. for operating revenue or share sales. The game will
end at the end of the current game turn.

Use whatever means necessary (e.g. additional banknotes/poker chips,
spreadsheet) to track additional cash flows until the end of the game.

\subsection{Final game turn}

When the last game turn indicated by the scenario is finished, the game is over.


\section{Forced sale of shares}

There are two situations in which the player is forced to sell stock
certificates.
\begin{itemize}
\item They hold shares beyond the certificate and/or ownership
  limits. In their next stock round turn, they must sell shares to
  comply with the limits.

\item If they are the president of a minor or public company which is
  obliged to buy a train, and their cash on hand combined with the
  company treasury is insufficient to do so. They must sell shares to
  raise the necessary cash. If all possible sales have been made and
  they still cannot afford a train, the player is bankrupt and the
  game ends immediately.
\end{itemize}

The following rules apply to all forced-sales of stock certificates.
\begin{itemize}
\item Shares cannot be sold that would cause a change of presidency of
  the company which caused the forced sale. However, sales can be made
  which would cause a change of presidency in other companies.

\item The president decides which company's stock certificates, in
  what order, and how many to sell.

\item The public company may be closed by the sale of stock
  certificates. In this case, all cash made here will be collected by
  the bank.

\item All shares to be sold must be sold together.

\item Open market limits, certificate limits, and ownership limits all apply.

\item The president can only sell shares for the amount necessary to
  purchase the train.
\end{itemize}

Even if the public company that caused the forced train purchase is closed
due to a forced sale, all cash prepared for payment of the train will
be seized by the bank. At this time, if the train is purchased, the
train will be put in the open market.


\newpage

\section*{Scenarios}

Several scenarios are available for selection depending on desired
play time and/or player skill.

\begin{enumerate}[label=\Alph*]
\item Hanshin Railways: 2 to 3 hours, 2 to 4 players
\item Keihanshin Railways: 2 to 3 hours
\item 1890: 3 to 5 hours
\item 1890 Simplified: 3 to 5 hours
\end{enumerate}

\newcounter{scenario}
\renewcommand*{\thesection}{\Alph{section}}
\renewcommand*{\theHsection}{scenarios.\Alph{section}}
\setcounter{secnumdepth}{3}
\setcounter{section}{0}

\section{Hanshin Railways Ver 1.4.1}

This scenario was designed to play as an 18XX introductory game. It is
designed to finish quickly with a small number of players, and almost
the same game phase structure. Experienced players can finish a game
in 2 to 3 hours.

Experience the battle of railways in Hanshin between Hanshin Railway,
Hankyu Railway, and JR.

In this scenario, 6-trains and D-trains are very unlikely to
appear. Once you get used to it, try playing the next `Keihanshin
Railways Scenario`.

\subsection{Number of players}

2 to 4 (2 to 3 recommended)

The two-player game is an austere experience whereas the three-player
game is more enjoyable. The four-player game is an unending
experience.

\subsection{Map}

The entire map west of the Yodo River (towards Kobe), including Osaka
Kita and Kyoto is used. As per the normal rules, track tiles may not be laid
pointing towards the Yodo River.

\subsection{Track tiles}

All usable (except for special tiles cities not included in the scenario).

\subsection{Trains}
\begin{tabular}{ll}
Type & Count \\
\hline
2T & 5\\
2+2T & Not used\\
3T & 4 \\
3+3T & Not used \\
4T & 3 \\
5T & 2 \\
6T & 1 \\
D & 5
\end{tabular}

\subsection{Money}
The bank size is 6,000円. Set aside all 500円 and 1000円
banknotes for use after the bank breaks. Divide 1,500円 evenly
among players as starting capital.

\begin{tabular}{|l|c|c|c|}
\hline
Players & 2 & 3 & 4 \\
\hline
Start Money & 750 & 500 & 375 \\
\hline
\end{tabular}

\subsection{Phases}
Phase 1 to phase 6 are played as normal. However, the concept of the
first half and second half of phases 1 and 2 is not relevant.

\subsection{Certificate Limits}
Use the following certificate limits according to the number of players:

\begin{tabular}{|l|c|c|c|}
\hline
Players & 2 & 3 & 4 \\
\hline
Certificates & 16 & 12 & 10 \\
\hline
\end{tabular}

\subsection{Companies}
Due to the limited scope of the map, fewer companies are used. There
are 5 private companies, 1 minor company and 4 public companies. Late
private companies are not used, except in the case where Kobe Electric
Railway converts to a late private company. Also, at the start of the
game, the initial stock round will start in the following company
order. Keishin Railway and Osaka Municipal Electric Railway use the par
value and the dividend amount as shown below.

\begin{table*}
\begin{tabular}{lllll}
 & Company & Par/Revenue (円) & Type & Modified? \\
\hline
A & Arima Railway & 20/5 & Private & \\
B & Kobe Tramway & 40/10 & Private & \\
C & Keishin Railway & 60/15 & Private & X \\
D & Hanshin Tramway & 110/20 & Private & \\
E & Osaka Municipal Electric Tramway & 120/30 & Private  & X \\
F & Kobe Electric Railway & 100/- & Minor  & \\
\end{tabular}
\end{table*}

\subsubsection{Private Companies}

In this scenario, every private company can be purchased by a public
company. Until the private company is purchased by a public company or
closed, the hexes with tracks for these private companies can not have
track tiles laid on them. The rules and benefits for each private
company are as follows:

\begin{description}
\item[Arima Railway] 20円 / 5円

  Once purchased by a public company, you can place an additional tile
  on Arima in addition to the regular track tile placement. The track
  tile does not need to be connected. This power disappears when Arima
  Railway closes.

\item[Kobe City Tramway] 40円 / 10円

  There are no special features.

\item[Keishin Railway]  60円 / 15円

  There are no special features.

\item[Hanshin National Highway Orbit] 110円 / 20円

  Comes with one share of Hanshin Electric Railway.
\end{description}

\subsubsection{Late private companies}

Late private companies are not used except for the case of exchanging
the Kobe Electric Railway minor company for a late private company.

\subsubsection{Minor Companies}
\begin{description}
\item[Kobe Electric Railway] \hfill

  Home: Tanigami \hfill Par Value: 100円  \hfill Starting Capital: 200円

  At any point during it's operating round turn, Kobe Electric Railway
  may convert to a late private company. After converting, station
  tokens are remvoed and all assets are returned to the bank. Although
  the Kobe Electric Railway late private company certificate does not
  count toward the certificate limit, its face value is 0円. From
  this point onwards, any public company may place a station token in
  the city previously occupied by Kobe Electric Railway's token
  (Tanigami).

\end{description}

\subsubsection{Public companies}
It is said that the development of the game is driven by public
companies. In order to ease play, no special rules are used for this
scenario.
\begin{description}
\item[Hanshin (Hanshin Electric Railway)] \hfill

  Home: Nishitomi \hfill Stations: 3

  The company is started when 4 shares are purchased because 1 share
  of Hanshin stock comes with the Hanshin Tramway private company.

  The Hanshin stock attached to the Hanshin Tramway can not be sold
  until the president stock of Hanshin is purchased.

\item[Hankyu (Hankyu Railways)] \hfill
Home: Toyonaka \hfill Stations: 4

  No special rules.

\item[JR (Japan Railways)] \hfill \label{JR}
Homes: Osaka Kita, Kyoto, Nara, Kobe \hfill Stations: 6

JR is treated as a regular public company except for the following rules:
\begin{itemize}
\item The par price must be 100円.
\item JR has a higher train limit than other public companies.

\begin{tabular}{ll}
  Phase & Limit \\
  \hline
  1 &  6 \\
  2-3 & 4 \\
  4-6 & 3
\end{tabular}
\item JR can only distribute operating revenue as half dividends. Half
  of the revenue (rounded up to the nearest 10円) is paid directly
  to the company treasury, the remaining half is distributed among
  shareholders. Revenue can also be withheld as normal.
\item In game phases 1 to 3, JR may place or upgrade two track tiles in
  two different hexes. Performing two tile actions on the same hex is
  not allowed.
\end{itemize}

\item[Sanyo (Sanyo Electric Railway)] Home: Akashi Stations: 2

No special rules.
\end{description}
\subsection{End of the game}

The game will end if any of the following conditions are met:
\begin{enumerate}
\item The bank breaks

\item Player bankruptcy

\item After 10 game turns.
\end{enumerate}

\subsection{Variants}

Please try to provide some variants. Also, try using the 1890 scenario
variant if you are used to it.

\subsubsection{Specialization of public companies}

Incorporate and play by the special rules of each railway company from the
1890 scenario.

\subsubsection{Start capital}

Reduce the amount of money held by the first player. The amount of
money should be decided between players.

\subsubsection{JR's rule change}

Play without special rules for JR (extra track actions, half
dividends, fixed par price).

\subsubsection{Number of game turns}

The game ends after 8 game turns instead of 10.

\newpage
\section{Keihanshin Railway Ver1.2}

This scenario is an extension to the Hanshin Railways
scenarios. Frequently, the 6-trains will not come out.

\subsection{Number of players}

2 to 4 players

\subsection{Map}

The entire map west of the Yodo River, and also the hexes directly
adjacent to the east (right) of the Yodo River. Osaka city includes
Osaka Kita and Osaka Higashi. The hex to the east of Kyoto is also
used.

\subsection{Track tiles}

All usable (except for unused special locations).

\subsection{Trains}

\begin{tabular}{ll}
Type & Count \\
\hline
2 & 5 \\
2+2 & Not used \\
3 & 4 \\
3+3 & Not used \\
4 & 3 \\
5 & 2 \\
6 & 1 \\
D & 6 \\
\end{tabular}

\subsection{Money}
The bank size is 6,000円. Set aside all 500円 and 1000円 for
use after the bank breaks.

Divide 1,500円 evenly among players as starting capital.

\begin{tabular}{|l|c|c|c|}
\hline
Players & 2 & 3 & 4 \\
\hline
Start Money & 750 & 500 & 375 \\
\hline
\end{tabular}

\subsection{Phases}
Phase 1 to phase 6 are played as normal. However, the concept of the
first half and second half of phases 1 and 2 is not relevant.

\subsection{Certificate Limit}
Use the following certificate limits according to the number of players:

\begin{tabular}{|l|c|c|c|}
\hline
Players & 2 & 3 & 4 \\
\hline
Certificates & 16 & 12 & 10 \\
\hline
\end{tabular}

\subsection{Companies}
Due to the limited scope of the map, fewer companies are used. There
are 5 private companies, 1 minor company and 5 public companies. Late
private companies are not used, except in the case where Kobe Electric
Railway converts to a late private company. Also, at the start of the
game, the initial stock round will start in the following company
order. Osaka Municipal Electric Railway uses the modified par
value and the dividend amount as shown below.

\begin{tabular}{lp{2cm}llc}
 & Company & Par/Revenue (円) & Type & Modified? \\
\hline
A & Arima Railway & 20/5 & Private  & \\
B & Kobe Tramway & 40/10 & Private  & \\
C & Hanshin Tramway & 110/20 & Private  & \\
D & Osaka Municipal Electric Tramway & 120/ 30 & Private  & X \\
E & Keishin Railway & 160/25 & Private & \\
F & Kobe Electric Railway & 100/- & Minor & \\
\end{tabular}

\subsection{Private Companies}

Same as the Hanshin Railways scenario. However, the following changes will
be made to Keishin Railway.

\begin{description}
\item[Keishin Railway]  160円/ 25円

Comes with one regular share of Keihan Electric Railway.
\end{description}

\subsection{Minor companies}
Same as the Hanshin Railways scenario

\subsection{Late private companies}
Same as the Hanshin Railways scenario

\subsection{Public companies}
Same as Hanshin Railways scenario, with the addition of Keihan Electric Railway.

\begin{description}
\item[Keihan (Keihan Electric Railway)] \hfill

Homee: Hirakata \hfill Stations: 3

The company will be started if 4 shares are purchased from the initial
offer, since the Keishin Railway private company comes with one share of
Keihan Electric Railway. No shares of Keihan Electric Railway may be
sold until its president's certificate has been purchased.
\end{description}

\subsection{End of the game}
The game ends if any of the following conditions are met,
\begin{enumerate}
\item The bank breaks
\item Player bankruptcy
\item After 8 game turns
\end{enumerate}

\subsection{B10 Variants}
Same with the Hanshin Railways scenario

\newpage
\section{1890 ver 3.1}

This scenario is the 1890 scenario.

If you are playing this game for the first time, the Hanshin Railways,
Keihanshin Railways, or Simplified 1890 scenario may be a better
introduction.

\subsection{Number of players}

The 1890 scenario is designed to be played with 2 to 7 players. A two
player game is very difficult and it will be difficult for a player
who makes two consecutive mistakes to be victorious. Three and four
players can enjoy various bargains if the player is familiar with the
rules. If you have fun as a multiplayer game, I recommend 5-6 play. It
will be the most standard play.

\subsection{Map}

The entire map is used.

\subsection{Track tiles}

All are available.

\subsection{Trains}

\begin{tabular}{ll}
Type & Count \\
\hline
2T & 9\\
2+2T & 3\\
3T & 5 \\
3+3T & 2 \\
4T & 4 \\
5T & 3 \\
6T & 2 \\
D & 6
\end{tabular}

\subsection{Money}
The bank size is 12,500円. Set aside 12 500円 and 12 1000円
banknotes for use after the bank breaks. Divide 2,520円 evenly
among players as starting capital.

\begin{tabular}{l|llllll}
Players & 2 & 3 & 4 & 5 & 6 & 7 \\ \hline
Money(円) & 1250 & 840 & 630 & 504 & 420 & 360
\end{tabular}

\subsection{Phases}

Phases 1 to phase 6 are played as normal.

\subsection{Certificate Limits}
Use the following certificate limits according to the number of players:

\begin{tabular}{l|llllll}
Players & 2 & 3 & 4 & 5 & 6 & 7\\
\hline
Certificates & 26 & 18 & 15 & 13 & 11 & 10 \\
\end{tabular}

\subsection{Companies}

In the 1890 scenario, 6 private companies, 5 minor companies, 4 late
private companies, and all 8 public companies are in use. At the start
of the game, start the initial stock round in the following company
order:

\begin{description}
\item[Private companies] \hfill

\begin{tabular}{lp{3cm}l}
 & Company & Par (円) \\
\hline
A & Arima Railway & 20\\
B & Kobe Tramway & 40\\
C & Hankai Tramway & 70 \\
D & Hanshin Tramway & 110\\
E & Keishin Railway & 160\\
F & Osaka Municipal Electric Railway & 120
\end{tabular}
\item[Minor companies] \hfill

\begin{tabular}{lll}
 & Company & Par (円) \\
\hline
G & Kanan Railway & 100\\
H & Osaka Electric Railroad & 200\\
I & Osaka Railway & 100 \\
J & Nara Electric Railway & 160\\
K & Kobe Electric Railway & 100
\end{tabular}
\end{description}

\subsection{Private companies}\label{privates}
Private companies block tile lays on the map hexes where their track
is. They also carry certain benefits and abilities which are explained
as follows:

\begin{description}
\item[Arima Railway] \hfill 20円/5円

  The public company which purchases Arima Railway may place a track
  tile in Arima in addition to its regular track tile placement. The
  tile does not need to be connected.

\item[Kobe Tramway] \hfill 40円/10円

  No special rules

\item[Hankai Tramway] \hfill 70円/ 15(5)円

  Does not close in phase 4. However, its revenue is reduced to 5円, it may
  no longer be purchased by a public company, and continues to count
  as one certificate toward the certificate limit.

\item[Hanshin Tramway] \hfill 110円/20円

Comes with one regular share of Hanshin Electric Railway.

\item[Keishin Railway] \hfill 160円/25円

Comes with one regular share Keihan Electric Railway.

\item[Osaka Municipal Electric Railway] \hfill 220円/ 40円 \label{osaka-municipal}

  The purchasing player immediate sets the par price of the Osaka
  Municipal Subway public company and receives the president's
  certificate. The par value of Osaka Muncipal Electric Railway then
  becomes 0円.

  During Osaka Muncipal Subway's first operating round turn, it may
  upgrade one of Osaka Kita, Osaka Higashi, or Osaka Nishi for free.

  Osaka Municipal Electric Railway closes when Osaka Municipal Subway
  buys a train. Even if Osaka Municipal Subway has started, but is
  trainless, Osaka Municipal Electric Railway will continue to stay
  open, but the public company's stock price will fall due to
  non-payment of dividends.
\end{description}

\subsection{Late private companies}

This scenario includes four late private companies. Their rules and
benefits are as follows:

\begin{description}
\item[Keifuku Electric Railroad] \hfill 200円/40円

  If Keihan has a station token in Kyoto, it will receive 40円 to
  its treasury each operating round.

\item[Kobe Rapid Transit ​​Railway] \hfill 240円/? \label{kobe}

  Kobe Rapid Transit Railway is a special railway company with no
  trains. There is no fixed income and it earns no revenue from its
  operation. Although it has a station token in Kobe, there is no
  obligation to own trains.

  Kobe Rapid Transit Railway is started at the moment of purchase by a
  player. Station tokens other than JR in Kobe will be returned to the
  owning public companies, the number 6 turn token will be placed in
  Kobe as a station token of Kobe Rapid Transit Railway. This station
  token occupies a city space like a regular station tokens and
  affects train operations in the same way. From this point on, if
  there is a free space in Kobe, any public company may place their
  station tokens in Kobe as usual.

  Public companies may ignore the existence of the Kobe Rapid Transit
  Railway station token by paying a 100円 fee to the bank. This
  payment is considered to be the company's one station token
  placement per operating round turn, but does not consume one of its
  station tokens. This does not apply to JR because it already has a
  station token in Kobe. Furthermore, Osaka Municipal Subway cannot
  use this privilege.

  Any public company that previously had its token in Kobe is
  considered to have obtained the privilege to ignore the Kobe Rapid
  Transit Railway token.

  Every time a railway company that does not have a station token in
  Kobe counts Kobe in its train route for revenue (including those
  companies which have obtained the privilege described above), Kobe
  Rapid Transit Railway pays half the value of Kobe as dividend,
  without changing the revenue for the other railway company.

\item[Kita-Osaka Kyuko Railway] \hfill 280円/60円

  In the operating round following the purchase of the first 6-train,
  pays a one-time dividend of 100円 in addition to the usual 60円 dividend.

\item[Semboku Rapid Railway] \hfill 320円/70円

  Any public company with a station token in Sakai receives 40円
  into its treasury each operating round.
\end{description}

\subsection{Minor companies}
\label{sec:minor-companies}

This scenario includes five minor companies. The number printed on the
minor company's charter is its operating order, and the corresponding
turn order token is used as its station token. Minor companies 2 and 3
operate out of the same station token in Osaka Higashi.

The minor companies 1 to 4 will convert or merge into the Kinki Nippon
Railways public company at certain points in the game. This conversion
or merger may be declared at any point during the minor company's
operating turn.

Each minor company owned by a player counts towards the player's
certificate limit. The rules for each minor company are as follows:

\begin{description}

\item[1 Kanan Railway] \hfill

Home: Norohara \\
Par price: 100円 \\
Start Capital: 100円

Beginning in phase 2, merger with Kinki Nippon Railway is
possible. The merger with Kinki is forced in phase 3. Kanan Railways
is closed and exchanged for a 10\% regular share of Kinki. Kinki
receives half of Kanan Railway treasury (rounded up) and its
trains. The remainder of the treasury is given to the owner of Kanan
Railway.

\item[2 Osaka Electric Railroad] \hfill

Home: Osaka Higashi \\
Par price: 200円 \\
Start capital: 200円

Osaka Electric Railroad is the predecessor of the Kinki Nippon
Railways public company. When purchased in the initial stock round,
the owner of Osaka Electric Railroad immediately sets the par price
for Kinki.

Beginning in phase 2, Osaka Electric Railroad may convert into Kinki
Nippon Railways. The conversion into Kinki is forced when the second
half of phase 2 begins. Transfer all of Osaka Electric Railroad's
assets to Kinki. The owner of Osaka Electric Railroad receives the
20\% president's certificate of Kinki in exchange, and Kinki
immediately floats.

\item[3 Osaka Railway] \hfill

Home: Osaka Higashi \\
Par price: 100 \\
Start capital: 100

When Kinki Nippon Railways floats, Osaka Railway is forced to
merge. All assets are transferred to Kinki, and the owner of Osaka
Railway receives a 10\% regular share of Kinki in exchange.

\item[4 Nara Electric Railway] \hfill

Home: Kyoto, Nara \\
Par price: 160円 \\
Start capital: 320円

May merge into Kinki Nippon Railways from the beginning of phase 4. Is
forced to merge at the beginning of phase 5. All assets are
transferred to Kinki and the owner of Nara Electric Railway receives 2
regular shares of Kinki in exchange.

If Kinki already has its station token in the same city as Nara
Electric Railway's tokens, the minor company's token is
removed. Otherwise, it is replaced with a Kinki station token.

\item[5 Kobe Electric Railway]\hfill

  Home: Tanigami \\
  Par price: 100円 \\
  Start capital: 200円

  At any point during its operating round turn, Kobe Electric Railway
  may convert to a late private company. After converting, station
  tokens are remvoed and all assets are returned to the bank. Although
  the Kobe Electric Railway late private company certificate does not
  count toward the certificate limit, its face value is 0円. From
  this point onwards, any public company may place a station token in
  the city previously occupied by Kobe Electric Railway's token
  (Tanigami).
\end{description}

\subsection{Public companies}

The 1890 scenario includes eight public companies. Each company has
specific rules:

\begin{description}
\item[JR (Japan National Railways)] \hfill

Homes: Osaka Kita, Kyoto, Nara, Kobe \\
Stations: 6

Historically, JR was a state-owned railway until phase 6, and it was
operated with the national interest under consideration. JR is played
as a normal public company except for the following:
\begin{itemize}
\item The par price for JR is 100円

\item JR has a higher train limit than other public companies.

Phase 1: 6 trains
Phase 2-3: 4 trains
Phase 4-6: 3 trains

\item When distributing operating revenue, JR pays half
  dividends. The first half of the revenue (rounded up) is paid to the
  company treasury. The remainder of the revenue is then distributed
  among shareholders. Revenue may also be withheld as usual.

\item In phases 1 to 3, JR can lay or upgrade 2 track tiles per
  operating round in different hexes. Performing two tile actions in
  the same hex in one operating round is prohibited.
\end{itemize}

\item[Osaka Subway (Osaka Municipal Subway)] \hfill \label{osaka-subway}

Home: Osaka Nishi \hfill Stations: 1

The par price for Osaka Municipal Subway is set immediately when the
Osaka Municipal Electric Railroad private company is purchased in the
initial stock round.

\paragraph{Subway service}
After phase 4, subway service is established in the city of
Osaka. Osaka Subway may ignore other companies' station tokens in the
Osaka tiles (Osaka Kita, Osaka Higashi, Osaka Nishi, and Osaka Minami)
if they have been upgraded to brown.

\paragraph{Special tile placement}
During its first operating turn, Osaka Subway can place one tile in
Osaka free of charge.

\item[Keihan (Keihan Electric Railway)] \hfill

Home: Hirakata \hfill Stations: 3

Keihan will be float when 4 shares have been bought from the
initial offer because the Keishin Railway private company comes with
one regular share of Keihan.

The Keihan share attached to Keishin Railway may not be sold until the
president's certificate of Keihan has been purchased.

\item[Nankai (Nankai Electric Railway)] \hfill

Home: Osaka Minami \hfill Stations: 3

No special rules

\item[Kinki (Kinki Nippon Railways)] \hfill
\label{kinki}

Home: Osaka Higashi (Kashiwara, Kyoto, Nara) \hfill Stations: 6

Kinki Nippon Railways was historically formed through the merger of a
large number of smaller railway companies. In this scenario, the Osaka
Electric Railroad, Osaka Railway, Kanan Railway, and Nara Electric
Railway minor companies will merge to form Kinki.

The par price of Kinki share certificates will be decided immediately
by the player who purchased Osaka Electric Railroad in the initial
stock round.

Each minor company will be replaced by one or two Kintetsu share
certificates upon merger with Kintetsu. Moreover, the president's
certificate of Kinki is reserved for Osaka Electric Railroad. These
trade-in share certificates are reserved by the bank until then and can not
be bought or sold until the merger or conversion. In other words, only
4 Kinki stock certificates can be purchased from the initial
offer. These four stock certificates can be bought and sold from the
first stock round as usual, but they will not earn any income until
Kinki begins operations.

Each minor company may declare merger or conversion into Kinki, in
between company operations. The process and timing for merger and
conversion is described in \autoref{sec:minor-companies}

Kinki Nippon Railway forms when Osaka Electric Railway declares that
it will convert. Then, the Osaka Railway minor company is immediately
merged into Kinki. Kanan Railway may also choose to merge at this
time. Kinki immediately floats, even if none of its shares have been
purchased from the initial offer. It now receives additional starting
capital equal to 4 times its par price.

Immediately after Kinki floats, it performs a special operating round
out of turn order. During this special operating round, the share
price of Kinki does not drop due to non-payment of dividends. If
dividends are paid, the share price increases as normal. Kinki does
not operate again during the round in which it performed this special
operating round.

Kinki's stock price will not rise until Nara Electric Railway merges,
as the stock certificates reserved for mergers do not count as being
sold.

If Kinki closes before all minor companies have merged, the remaining
minor companies are unable to merge with Kinki and will close when they would
otherwise be forced to merge.

\item[Sanyo (Sanyo Electric Railway)] \hfill

Home: Akashi \hfill Stations: 2

No special rules

\item[Hanshin (Hanshin Electric Railway)] \hfill

Home: Nishinomiya \hfill Stations: 3

Hanshin floats when 4 shares are purchased, since the Hanshin Tramway
private company comes with 1 share of Hanshin stock.

Until the Hanshin president's certificate is purchased, the share
attached to Hanshin Tramway cannot be sold.

\paragraph{Hanshin Tigers}
Hanshin Electric Railway receives bonus revenue due to the popularity
of the Hanshin Tigers baseball team. After Nishinomiya is upgraded to
brown, if Hanshin operates routes which include Nishinomiya, increase
operating revenue by 10円, and also add 10円 to the Hanshin company
treasury from the bank.

\item[Hankyu (Hankyu Railways)] \hfill

Home: Toyonaka \hfill
Stations: 4

\paragraph{Takarazuka Revue}
Hankyu profits from popularity of the Takarazuka Revue musical
troupe. If Hankyu has a station token in Takarazuka, it receives
40円 to its treasury each operating round.


\paragraph{Hankyu Land Commercial Code}
Every time Hankyu lays a yellow track tile, it receives 10円 into the
company treasury.

\end{description}

\subsection{End of the game}

The game will end if any of the following conditions are met:
\begin{enumerate}
\item The bank breaks

\item Player bankruptcy

\item After 10 game turns.
\end{enumerate}

\subsection{Variants}
These variants can be used to change the evolution of the game. Some
variants will disrupt game balance and change the experience
significantly.

\subsubsection{Hidden personal cash (recommended)}
The amount of player cash is secret. Players do not need to respond to
inquiries about their cash holdings.

\subsubsection{Hiding of train purchase price}
The buying and selling of trains between companies must be made public
to other players, but it is not necessary to make their prices public.

\subsubsection{Change D-trains to 8-trains}
D-trains can visit only 8 cities in their route. All other properties,
including the ability to trade in a 4-, 5-, or 6-train, are the same.

\subsubsection{Forced closure of minor companies}
At the beginning of phase 5, Kobe Electric Railway is forced to
convert to a late private company.

\subsubsection{End of game when highest stock price is reached}
The game is over when the stock price of a public company reaches 400円.
If the price reached 400円 during the operating round, the game ends
immediately after the company's operating turn. If the prices reaches
400円 at the end of a stock round, the game ends immediately.

\subsubsection{Kobe Rapid Transit Revenue Control}
Reduce the value of Kobe to 10円 in yellow, 15円 in green, and 20円 in brown.

\subsubsection{No late private company abilities}
Remove all late private company special abilities. Kobe Rapid Transit
Railway pays a fixed dividend of 50円.

\subsubsection{Track tile placement restrictions}
Brown tile \#78 may only be placed in Nara or a city bordering the sea
or the Yodo river.

\subsubsection{Public company floatation}
This variant changes the rules for floating public companies, making
them easier to start. Incorporating this variant will produce a more
aggressive game experience.

When a company would begin its very first operating turn, check the
number of 10\% shares which have been sold from the initial
offering. If this number is greater than or equal to the rank of the
train currently for sale from the bank, then the company floats and
may operate. Otherwise, the company is considered not to have floated,
and will not take an operating turn. At the beginning of the game,
only the 20\% president's certificate needs to be sold in order for a
company to float and operate. Later in the game, if for example the
current train available for sale is a 3T or 3+3T, then a public
company must have sold at least 3 shares from the initial offer in
order float and operate.

Instead of receiving ten times the par value when floating, public
companies receive money as shares are sold from the initial offer. The
shares of Hanshin, Keihan, and Osaka Subway that are attached to
private companies are considered to be already sold, and the
respective public companies will begin the game with those funds in
their treasuries.

Additionally:
\begin{itemize}

\item After phase 5, public companies will be started using the normal
  rules. 50\% of shares must be sold from the initial offer to float,
  and the company receives 10 times the par value to its treasury.

\item JR does not float unless 5 shares (50\%) are purchased
  from the initial offer

\item Kinki Nippon Railway is started as normal, when Osaka Electric
  Railway converts. If the non-reserved shares of Kinki are bought
  from the initial offer before the company is formed, the money from
  those sales is still added to the Kinki treasury.

\end{itemize}

\subsubsection{Changes in conversion from JNR to JR}
Historically, JR transformed from a national railway to a private
railway company. After phase 6, JR will pay full dividends instead of
half.

\subsubsection{Conversion of Kintetsu to a normal company}
While the treatment of Kinki Nippon Railways in 1890 aims to simulate
its historical development, it also creates extra rules burdens. In
this variant, minor companies 1-4 are eliminated, and Kinki is started
like a normal public company.

In addition to 2 regular station tokens, Kinki will have 4 station
tokens that are automatically placed at the beginning of the following
game phases:

\begin{description}
\item[Phase 1] Osaka Higashi
\item[Phase 2] Kashiwara
\item[Phase 3] Nara
\item[Phase 4] Kyoto
\end{description}

Automatically placed station tokens are placed at the very beginning
of each operating round, at no cost. These tokens are treated as home
tokens, and reserve a space in their cities.  Follow normal token
placement rules for the automatic station tokens: connectivity and a
free city space is required. If possible, multiple automatic tokens
may be placed at once.

When using this variant, remove three 2-trains and all 3+3-trains.

\subsubsection{Hanshin Tigers}
This variant requires one six-sided die. After Nishinomiya is upgraded
to brown, instead of receiving a fixed amount, Hanshin Electric
Railway will receive a variable bonus if its routes include
Nishinomiya.

Each operating turn in which Hanshin runs to the brown Nishinominya
tile, roll the die. The value of Nishinomiya will be determined by the
result and the following table:

\begin{tabular}{l|llllll}
Die Roll & 1 & 2 & 3 & 4 & 5 & 6 \\
\hline
Value & 100 & 60 & 50 & 40 & 40 & 40 \\
\end{tabular}

\subsubsection{The Great Hanshin Earthquake}
This variant simulates the Great Hanshin Earthquake of January 1995,
which caused major damage to the Hanshi region. In particular,
Nishinomiya, Ashiya, and Kobe suffered heavy damage, and urban
functions were destroyed in the city centers. Building collapse and
fires caused many to lose their homes and many of the affected people
still live in refugee housing. Many industries were disrupted due to
the ensuing financial instability and economic recession. Despite
signficant time and money spent for recovery, the effects of this
disaster are still seen today.

Label the reverse side of one D-train with the word ``Earthquake'',
and shuffle it with the other D-trains during game setup. The Great
Hanshin Earthquake will occur when this labeled D-train is purchased,
unless the bank has already broken at this point.

If the earthquake occurs, then until the next stock round Hanshin
Railway's income will be 0, and the incomes for Hankyu, Sanyo, and JR
will be reduced by half. During this time, stock prices will be frozen
and will not move.

Kobe Rapid Transit Railway is closed, and the station tokens in Kobe
and Ashiya are removed and returned to their owning companies. The
Nishinomiya tile is downgraded one level. Tokens in Itami are flipped
face down. After the next stock round, companies may pay 100円 to flip
them back face up.

Charity Payments: Each player pays the following amount to the bank by
the end of the next stock round.

\begin{align*}
&(\textrm{\# D-trains already sold} \times \textrm{100円}) + \\
&(\textrm{\# of shares held} \times \textrm{200円})
\end{align*}

This amount is calculated at the time the earthquake occurs. If cash
on hand is insufficient to make the payment, then shares must be sold to
make up the difference.


\subsubsection{Second World War Air Raids}

At the end of World War II, the United States military conducted a
sustained campaign of air raids in cities around Japan's four
largest industrial areas. These bombings resulted in many civilian
deaths and impeded economic recovery after the war. The Hanshin region
was particularly impacted as it formed a large part of Japan's
industrial base, Osaka, Kobe, and Amagasaki were targeted for bombing.

After the last 3+3-train is purchased, the air raids will take
place. Discard all green tiles from Kobe, Ashiya, Nishinomiya,
Amagasaki, Osaka Kita, Osaka Nishi, Osaka Higashi, and Osaka Minami,
as well as surrounding hexes. Yellow tiles will not be
discarded. Green track tiles that were removed from empty map hexes
must be laid again starting with yellow, and repaying any terrain costs.

After green tiles have been removed, station tokens are
replaced. Tokens may not be replaced into spaces reserved for
unstarted companies. If a city contains station tokens from multiple
companies and there are no longer enough token spaces, then the tokens
are placed in operating order beginning with the company after the one
that purchased the last 3+3-train. Remaining tokens will be placed on
the tile, as a reserved token. When the tile is upgraded creating a
new token space, the reserved token will automatically fill that
space. At any time, reserved tokens may be removed to be used
elsewhere as a normal 100円 station token.

\end{multicols}

\renewcommand*{\thesection}{\Alph{section}}
\renewcommand*{\theHsection}{appendix.\Alph{section}}

\section{Phase Summary}

\definecolor{yellow}{HTML}{BD9704}
\definecolor{green}{HTML}{4B7521}
\definecolor{brown}{HTML}{753B0B}

\begin{tabular}{|l|l|l|l|p{6cm}|l|l|}
\hline
Phase & \parbox[t]{2cm}{Train\\(Quantity)} & Rust & \parbox[t]{3cm}{Train Limit\\(Minor/Public/JR)} & Comments & ORs & Tiles \\ \hline
1 & 2T (9) & & \multirow{4}{*}{2/4/6} & & \multirow{2}{*}{1} & \cellcolor{yellow}\\ \cline{1-3} \cline{5-5}
1$\frac{1}{2}$ & 2+2T (3) & & & & & \cellcolor{yellow}\\ \cline{1-3} \cline{5-6}
2 & 3T (5) & & &
\begin{itemize}[nosep, left=0pt]
  \item Public companies may buy private companies
  \item Osaka Electric \emph{may} convert
  \item Kanan \emph{may} merge
\end{itemize}
& \multirow{3}{*}{2} & \cellcolor{green} \\ \cline{1-3} \cline{5-5}
2$\frac{1}{2}$ & 3+3 (2) & & &
\begin{itemize}[nosep, left=0pt]
\item Osaka Electric \emph{must} convert
\end{itemize}
& & \cellcolor{green} \\ \cline{1-5}
3 & 4T (4) & 2T & 1/3/4 &
\begin{itemize}[nosep, left=0pt]
\item Kanan \emph{must} merge
\end{itemize}
& & \cellcolor{green} \\ \cline{1-6}
4 & 5T (3) & 2+2T & \multirow{3}{*}{1/2/3} &
\begin{itemize}[nosep, left=0pt]
\item Private companies close
\item Nara Electric \emph{may} merge
\end{itemize}
& \multirow{3}{*}{3} & \cellcolor{brown} \\ \cline{1-3} \cline{5-5}
5 & 6T (2) & 3T, 3+3T & &
\begin{itemize}[nosep, left=0pt]
\item Nara Electric \emph{must} merge
\item Bonus dividend for Kita-Osaka Kyuko Railway
\end{itemize}
& & \cellcolor{brown} \\ \cline{1-3} \cline{5-5}
6 & D (6) & 4T & & & & \cellcolor{brown} \\ \hline
\end{tabular}

\section{Frequently Asked Questions}
This is a list follows the rubric used by the 18xx Rules Differences
website
(\url{http://www.fwtwr.com/18xx/rules_difference_list/index.htm})

\renewcommand*{\thesubsection}{\arabic{subsection}}

\subsection{First Share Dealing Round}

\subsubsection{How much cash do players start with?}
\begin{tabular}{l|llllll}
Players & 2 & 3 & 4 & 5 & 6 & 7 \\ \hline
Money(円) & 1250 & 840 & 630 & 504 & 420 & 360
\end{tabular}

\subsubsection{Does the price of a private company drop by 5 for no
  sale in the first round?}
Yes, first private only. (\autoref{isr-arima})

\subsubsection{Can you sell company shares in the first round?}
No. (\autoref{sr-sell-restrictions})

\subsubsection{Can you make advance bids?}
Yes, at 5 or more over par and over any other bid. (\autoref{isr-bidding})

\subsection{Subsequent share dealing rounds}

\subsubsection{Is there a specific order to buying and selling on your turn?}
Sell then buy, \emph{or} buy then sell. (\autoref{sr-sell-buy})

\subsubsection{Are you limited to buying one certificate on your turn?}
Yes. (\autoref{sr-buy})

\subsubsection{When can you first sell shares in a company?}
From the second stock round onwards. (\autoref{sr-sell-restrictions})

\subsubsection{Does the bank pool have a per-company share limit?}
50\% (\autoref{sr-sell-restrictions})

\subsubsection{What are the player certificate limits?}
\begin{tabular}{l|llllll}
Players & 2 & 3 & 4 & 5 & 6 & 7\\
\hline
Certificates & 26 & 18 & 15 & 13 & 11 & 10 \\
\end{tabular}

\subsubsection{What are the player certificate limits for shares in one company?}
60\%. Shares in the brown zone do not count towards this. (\autoref{sec:ownership-limit-exceptions})

\subsubsection{Does the stock price drop when stock is sold?}
Yes, 1 row per share. (\autoref{sr-sell-price-drop})

\subsubsection{Does the stock price go up at the end of the share dealing round for a fully-held corporation?}
Yes. (\autoref{sr-sold-out})

\subsubsection{Can you buy a certificate and immediately sell a certificate in the samecompany?}
Yes. (\autoref{sr-sell-buy})

\subsubsection{Can companies buy shares?}
Privates, in ORs, once a 3 train has been sold, and only public
companies. (\autoref{or-private-purchase})

\subsubsection{What ends a share dealing round?}
Each player consecutively not making a purchase or a sale. The
priority then goes to the player after the one who last made a
purchase or a sale. (\autoref{sr-end-of-sr})

\subsection{Company flotation}

\subsubsection{Do you lay the base station token immediately upon floating?}
The token is placed on the home base at the start of the operating
round in which it will first run. (\autoref{or-home-station})

\subsubsection{How many shares must be sold for a company to float?}
Minor companies float when sold in the ISR. Public companies float
when 50\% of shares are sold from the IPO, except for Kinki Nippon
Railways, which floats when a minor company converts into
it. (\autoref{sec:starting-companies})

\subsubsection{Does a company get full capitalization upon floating?}
Yes. (\autoref{floating-initial-capital})

\subsubsection{How is a share company's initial (par) price determined?}
The price is set by the player who purchases the President's
certificate, choosing a value selected from a range of predefined
prices. The purchaser of the Osaka Electric Railway minor company sets
the par price for Kinki. (\autoref{floating-par-price})

\subsection{Operations}

\subsubsection{In what order do companies operate?}
First, minor companies in numeric order, then public companies in
descending share price. If two companies occupy the same space, the
company with its token on top operates first. If two companies have
the same price but in different spaces, the company with its token
furthest to the right operates first. (\autoref{or-operating-order})

\subsubsection{If you sell shares so that their tokens end up in one
  stack, what order are they stacked in?}
Unspecified.

\subsection{Tile Lays}

\subsubsection{Where can you make an initial tile lay?}
As for 1830. (\autoref{or-laying-tiles})

\subsubsection{Can you lay two tiles in a turn?}
Only by using the special property of a private. (\autoref{privates})

\subsubsection{Must a tile replacement extend exiting track?}
Not specified, presumably permissive.

\subsubsection{Do villages upgrade?}
No.

\subsection{Station Markers}

\subsubsection{Cost of station markers}
0 for home bases, 40, 100.

\subsubsection{Can you lay more than one station marker per turn?}
No. (\autoref{or-normal-stations})

\subsubsection{Where can you lay a station marker?}
The newly-laid station marker must be reachable from one of the laying
company's existing station markers by an arbitrarily large
train. (\autoref{or-station-placement-requirements})

\subsubsection{When is a company's first station marker laid?}
When it first operates. (\autoref{or-home-station})

\subsection{Train Runs}

\subsubsection{Can you run into a city completely filled by rival station markers?}
Yes. (\autoref{or-route-definition})

\subsubsection{Can you do a run that passes through a city completely
  filled by rival station markers?}
No, except for Kobe Rapid Transit Railway, and Osaka Subway in
brown. (\autoref{or-route-definition}, \autoref{kobe},
\autoref{osaka-subway})

\subsubsection{Unusual rules about running}
Plus trains (2+2T, 3+3T) can include an additional number of villages
in their run.

\subsubsection{Can one train run to two stations on the same tile?}
Yes. (\autoref{or-route-definition})

\subsubsection{Is double-heading allowed?}
No.

\subsubsection{Rules about villages}
No special rules.

\subsubsection{Must the maximum possible revenue be claimed?}
No. (\autoref{or-route-choice})

\subsection{Payment of Earnings}

\subsubsection{Does stock move right for payment of dividends?}
Yes. (\autoref{or-moving-stock-prices})

\subsubsection{What dividend payments go into the company's treasury?}
Those for shares in the bank pool. (\autoref{or-public-dividends})

\subsubsection{Does stock move left for withheld earnings?}
Yes. (\autoref{or-moving-stock-prices})

\subsubsection{Can a company make a partial payout?}
Minor companies and JR always make a 50\%
payout. (\autoref{or-minor-dividends}, \autoref{JR})

\subsection{Purchasing Trains}

\subsubsection{Can companies buy trains from one another?}
Yes, minimum price 1円. (\autoref{trains-buying-from-companies})

\subsubsection{Must a major share company buy a train if it does not have one?}
If it has a route. (\autoref{train-obligation})

\subsubsection{Can trains be sold back to the bank?}
No.

\subsubsection{When a company is forced to buy a train and connot buy one
  with its own means, what train may it then buy?}
The cheapest train from the open market or the bank. (\autoref{train-force-buy})

\subsubsection{Must a minor company buy a train if it does not have one?}
If it has a route.

\subsubsection{Can trains of the final type be purchased as soon as one
  train of the next-to-last type is purchased?}
Yes. (\autoref{sec:buying-trains})

\subsubsection{Can a company buy more than one train from the bank per OR?}
Yes.

\subsection{Private Companies}

\subsubsection{Are private companies purchaseable between players?}
Yes, on the buyer's turn. (\autoref{sr-buy})

\subsubsection{Are private companies purchaseable by share companies}
Yes. Public companies may purchase privates after a 3-train has been
sold, at from half to twice face value. (\autoref{or-private-purchase})

\subsubsection{Does a private company prevent builds in its home hex(es)
  while it is owned by a player?}
Yes.

\subsubsection{Does using a private company's special property close it?}
Only Osaka Municipal Electric Railway, when Osaka Subway buys its first train.

\subsubsection{When do private companies close?}
When the first 5-train is sold.

\subsubsection{Can you buy and sell private companies in other ways?}
Late private companies may be bought from the bank from SR2 onwards,
but may not be sold to other players or companies.

\subsection{Directorship/Presidency of a Share Company}

\subsubsection{Can you sell the director's certificate into the bank pool?}
No.

\subsubsection{Can you exchange the director's certificate for regular
  shares from another player when you sell shares to the bank pool?}
Yes.

\subsubsection{After a sale of shares forces a change in Director, who
  gets it in case of a tie?}
Next qualifying player on seller's left. (\autoref{shares-change-president})

\subsection{Game Phases}
See \autoref{phase-summary}

\subsection{End of Game}

\subsubsection{Game ends immediately with a bankruptcy?}
Yes. (\autoref{endgame-player-bankruptcy})

\subsubsection{What happens if the bank runs out of money during an
  operating round?}
Game ends at the end of current set of operating
rounds. (\autoref{endgame-bank-break}

\subsubsection{What happens when stock first hits the top end of the market?}
Nothing.

\subsubsection{What happens if the bank runs out of money during a stock
  round?}
Complete the next set of operating rounds. (\autoref{endgame-bank-break})

\subsection{Secrecy}

\subsubsection{Is a player's cash secret, or open for inspection?}
Open. (\autoref{player-cash})

\subsubsection{Is a company's cash secret, or open for inspection?}
Secret. (\autoref{company-charters})

\subsubsection{Is the cash involved in transactions secret, or must it be
  made public?}
Not stated.

\subsubsection{Is the cash in the bank secret, or must it be made public?}
Not stated. Assumed to be public.

%%% Local Variables:
%%% mode: latex
%%% TeX-master: "1890rules-en-translation"
%%% End:

\end{CJK}
\end{document}
